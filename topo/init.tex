\section{Topologie initiale}

\begin{Def}
Soient $I$ un ensemble quelconque non vide, $(Y_i)_{i\in I}$ une famille d'espaces topologiques, $X$ un ensemble et pour tout $i\in I, f_i:X\maps Y_i$. On appelle topologie initiale (sur $X$) la topologie la plus faible qui rende continues toutes les $f_i$, pour $i\in I$.
\end{Def}

\begin{Prop}
Les topologies faible, pr�faible et produit sont des topologies initiales.
\end{Prop}
\begin{proof}
Pour la topologie faible, prendre $I=E\dual$ et pour tout $x\dual\in I$, $f_{x\dual} = x\dual$ et $Y_{x\dual}=\mathbb{K}$ muni de la topologie usuelle. Pour la topologie pr�faible, prendre $I=E$ et pour tout $x\in I$, $f_{x}=ev_x$ et $Y_{x}=\mathbb{K}$ muni de la topologie usuelle.
\end{proof}

La topologie produit, qui est d�finie comme suit, est clairement une topologie initiale.

\begin{Def}[Topologie produit]\label{topoprod}
Soient $I$ un ensemble non vide, et $(Y_i)_{i\in I}$ des espaces topologiques. On note $X=\prod\limits_{i\in I}Y_i$, et un �l�ment $x\in X$ se note $(x_i)_{i\in I}$, o� $x_i\in Y_i$, pour tout $i\in I$. On d�finit, pour tout $i\in I$, les applications $\pi_i:X\maps Y_i:x\mapsto x_i$, appel�es projections. La topologie sur $X$ est la moins fine qui rende continues toutes les projections, est appel�e topologie produit et est not�e $\Tau_\pi$.
\end{Def}

\begin{Prop}\label{proptopoinit}
On munit $X$ de la topologie initiale. Soit $Z$ un espace topologique et $\varphi:Z\maps X$. Alors on a l'�quivalence suivante:
\begin{itemize}
\item $\varphi$ est continue
\item $\forall i\in I, \varphi\circ f_i$ est continue
\end{itemize}
\end{Prop}

