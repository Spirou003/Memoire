\chapter{Bases de Schauder}\label{chapter_schauder}

Dans le chapitre pr�c�dent, on a vu une propri�t� dont la d�monstration n�cessitait le raisonnement suivant:\newline

\begin{itemize}
\item On se fixe une base alg�brique de l'espace vectoriel $E$.
\item On prend un point de $E$ et une suite qui converge vers ce point.
\item On d�compose chaque vecteur choisi suivant la base fix�e.
\item Puisque la suite est convergente, les coordonn�es sont elles aussi convergentes.\newline
\end{itemize}
Bien que ce type de raisonnement soit toujours vrai en dimension finie, ce n'est plus le cas en dimension infinie. Par exemple, pla�ons-nous dans $l^\infty$. Les $(e_n)_{n\in\mathbb{N}}$ forment une famille libre de vecteurs de $l^\infty$, mais ne forment pas une base alg�brique. La suite $x=\left(\frac{1}{n}\right)_{n\in\mathbb{N}_0}$ est lin�airement ind�pendante des $(e_n)_{n\in\mathbb{N}}$ pour des combinaisons lin�aires finies. Alors, on peut utiliser le th�or�me de la base incompl�te afin d'obtenir une base alg�brique de $l^\infty$ qui contient les $(e_n)_{n\in\mathbb{N}}$ ainsi que $x$. On consid�re maintenant la suite $(x^{(n)})_{n\in\mathbb{N}}$ telle que pour $n\in\mathbb{N}_0$, on a

\[
x^{(n)}
=
\sum_{m=1}^n \frac{e_m}{m}
\]

Que l'on peut repr�senter

\[
\begin{array}{ccccccccccccccccccc}
x^{(1)} & = & ( & 1 & , & 0 & , & 0 & & ... & & ... & & 0 & , & 0 & & ... & )\\
x^{(2)} & = & ( & 1 & , & 1/2 & , & 0 & & ... & & ... & & 0 & , & 0 & & ... & )\\
x^{(3)} & = & ( & 1 & , & 1/2 & , & 1/3 & & ... & & ... & & 0 & , & 0 & & ... & )\\
\vdots &  &  & \vdots &  & \vdots &  & \vdots &  & \ddots &  &  &  & \vdots &  & \vdots & & & \\
\vdots &  &  & \vdots &  & \vdots &  & \vdots &  &  &  & \ddots &  & \vdots &  & \vdots & & & \\
x^{(n)} & = & ( & 1 & , & 1/2 & , & 1/3 & & ... & & ... & , & 1/n & , & 0 & , & ... & )\\
\vdots &  &  & \vdots &  & \vdots &  & \vdots &  &  &  &  &  & \vdots &  & \vdots & & & \\
\end{array}
\]

On a bien que chaque $x^{(n)}$ appartient � $l^\infty$ et que $x^{(n)}\rightarrow x$. Cependant on n'a pas de convergence en coordonn�es de la base alg�brique choisie. En effet, $x$ est sa propre d�composition dans la base choisie, et quel que soit $n$, la d�composition de $x^{(n)}$ ne fait pas intervenir $x$.\newline

On va donc d�finir un nouveau type de base pour lequel la convergence g�n�rale donne la convergence des coordonn�es. Ensuite, nous �tudierons ses liens avec le dual et le bidual.

Dans le chapitre pr�c�dent, on a vu une propri�t� dont la d�monstration n�cessitait le raisonnement suivant:\newline

\begin{itemize}
\item On se fixe une base alg�brique de l'espace vectoriel $E$.
\item On prend un point de $E$ et une suite qui converge vers ce point.
\item On d�compose chaque vecteur choisi suivant la base fix�e.
\item Puisque la suite est convergente, les coordonn�es sont elles aussi convergentes.\newline
\end{itemize}
Bien que ce type de raisonnement soit toujours vrai en dimension finie, ce n'est plus le cas en dimension infinie. Par exemple, pla�ons-nous dans $l^\infty$. Les $(e_n)_{n\in\mathbb{N}}$ forment une famille libre de vecteurs de $l^\infty$, mais ne sont pas une base alg�brique. La suite $x=\left(\frac{1}{n}\right)_{n\in\mathbb{N}_0}$ est lin�airement ind�pendante des $(e_n)_{n\in\mathbb{N}}$ pour des combinaisons lin�aires finies. Alors, on peut utiliser le th�or�me de la base incompl�te afin d'obtenir une base alg�brique de $l^\infty$ qui contient les $(e_n)_{n\in\mathbb{N}}$ ainsi que $x$. On consid�re maintenant la suite $(x^{(n)})_{n\in\mathbb{N}}$ telle que pour $n\in\mathbb{N}_0$, on a

\[
x^{(n)}
=
\sum_{m=1}^n \frac{e_m}{m}
\]

Que l'on peut repr�senter

\[
\begin{array}{ccccccccccccccccccc}
x^{(1)} & = & ( & 1 & , & 0 & , & 0 & & ... & & ... & & 0 & , & 0 & & ... & )\\
x^{(2)} & = & ( & 1 & , & 1/2 & , & 0 & & ... & & ... & & 0 & , & 0 & & ... & )\\
x^{(3)} & = & ( & 1 & , & 1/2 & , & 1/3 & & ... & & ... & & 0 & , & 0 & & ... & )\\
\vdots &  &  & \vdots &  & \vdots &  & \vdots &  & \ddots &  &  &  & \vdots &  & \vdots & & & \\
\vdots &  &  & \vdots &  & \vdots &  & \vdots &  &  &  & \ddots &  & \vdots &  & \vdots & & & \\
x^{(n)} & = & ( & 1 & , & 1/2 & , & 1/3 & & ... & & ... & , & 1/n & , & 0 & , & ... & )\\
\vdots &  &  & \vdots &  & \vdots &  & \vdots &  &  &  &  &  & \vdots &  & \vdots & & & \\
\end{array}
\]

On a bien que chaque $x^{(n)}$ appartient � $l^\infty$ et que $x^{(n)}\rightarrow x$. Cependant on n'a pas de convergence en coordonn�es de la base alg�brique choisie. En effet, $x$ est sa propre d�composition dans la base choisie, et quel que soit $n$, la d�composition de $x^{(n)}$ ne fait pas intervenir $x$.\newline

On va donc d�finir un nouveau type de base pour lequel la convergence g�n�rale donne la convergence des coordonn�es. Ensuite, nous �tudierons ses liens avec le dual et le bidual.

\section{Suite basique, base de Schauder, constante de base}

\begin{Def}
Soient $E$ un espace vectoriel norm� et $(e_n)_{n\in\mathbb{N}}$ une suite de vecteurs de $E$. On dira que $(e_n)_{n\in\mathbb{N}}$ est une suite basique si pour tout $x\in\SpanAdh((e_n)_{n\in\mathbb{N}})$, il existe une unique suite de scalaires $(x_n)_{n\in\mathbb{N}}$ telle que
\[\sum_{n\in\mathbb{N}}x_n e_n=x\]

Si de plus, $\SpanAdh((e_n)_{n\in\mathbb{N}})=E$, on dit que c'est une base de Schauder, et on appelle $x_n$ la $n^\text{i�me}$ coordonn�e de $x$.
\end{Def}

\subsubsection*{Exemples}
\begin{itemize}
\item La base canonique $(e_n)_{n\in\mathbb{N}}$ est une base de Schauder des espaces $c_0$ et $l^p$, pour $p\in[1,+\infty[$.
\item L'espace $l^\infty$ n'�tant pas s�parable, il ne contient aucune base de Schauder. Pour rappel, un espace vectoriel norm� est s�parable s'il admet une partie d�nombrable dense.
\end{itemize}

\begin{Lemme}\label{espseriescgebanach}
Soit $E$ un espace de Banach qui contient une base de Schauder $(e_n)_{n\in\mathbb{N}}$. L'espace vectoriel
\[
F :=
\left\{
(a_n)_{n\in\mathbb{N}}\in\mathbb{K}^\mathbb{N}
~|~
\sum_{n\in\mathbb{N}}a_n e_n\text{ converge}
\right\},
\]
muni de la norme
\[
\norme{(a_n)_{n\in\mathbb{N}}}_F
=
\sup_{N\in\mathbb{N}}\norme{\sum_{n=0}^N a_n e_n}
\]
est un espace de Banach.
\end{Lemme}
\begin{proof}
Le fait que $(F,\norme{.}_F)$ soit un espace vectoriel norm� est �vident. On montre qu'il est complet. Soit une suite de Cauchy $(a^{(n)})_{n\in\mathbb{N}}$ de vecteurs de $F$. On a que
\[
\forall\varepsilon>0,\exists n_0\in\mathbb{N}\text{ tel que }\forall n\geqslant m\geqslant n_0,
\norme{a^{(n)}-a^{(m)}}_F=\sup_{N\in\mathbb{N}}\norme{\sum_{k=0}^N (a^{(n)}_k-a^{(m)}_k)e_k}<\varepsilon
\]
d'o�
\[
\forall N\in\mathbb{N},\forall\varepsilon>0,\exists n_0\in\mathbb{N}\text{ tel que }\forall n\geqslant m\geqslant n_0,
\norme{\sum_{k=0}^N (a^{(n)}_k-a^{(m)}_k)e_k}<\varepsilon
\]
Cela signifie que pour tout $N\in\mathbb{N}$, la suite $\left(\sum\limits_{k=0}^N a^{(n)}_k e_k\right)_{n\in\mathbb{N}}$ est de Cauchy. Elle converge donc vers un �l�ment, que l'on note $y^{(N)}$. Puisque $y^{(N)}$ est une limite de combinaisons lin�aires de $e_0,...,e_N$, il existe $y^{(N)}_0,...,y^{(N)}_N\in\mathbb{K}$ tels que $y^{(N)}=\sum\limits_{k=0}^N y^{(N)}_k e_k$. De plus, $(e_n)_{n\in\mathbb{N}}$ est une base de Schauder de $E$, donc on peut identifier
\[
\sum_{k=0}^N \lambda_k e_k \longleftrightarrow (\lambda_0,...,\lambda_N)
\]
D'o� $(a^{(n)}_0,...,a^{(n)}_N)\rightarrow(y^{(N)}_0,...,y^{(N)}_N)$, c'est-�-dire que pour $k\in\{1,...,N\}$, $(a^{(n)}_k)_{n\in\mathbb{N}}$ est convergente. On note alors $a_k$ sa limite. Puisque ce raisonnement est valable quelque soit $N$, on obtient une suite $a=(a_k)_{k\in\mathbb{N}}$. Il nous reste � d�montrer que $a\in F$ et que $a^{(n)}\rightarrow a$.

Comme $(a^{(n)})_{n\in\mathbb{N}}$ est de Cauchy, on a que
\[
\forall\varepsilon>0,\exists n_0\in\mathbb{N}\text{ tel que }\forall n\geqslant m\geqslant n_0,\forall N\in\mathbb{N},
\norme{\sum_{k=0}^N (a^{(n)}_k-a^{(m)}_k)e_k}<\varepsilon
\]
Par passage � la limite sur $m$, on obtient
\[
\forall\varepsilon>0,\exists n_0\in\mathbb{N}\text{ tel que }\forall n\geqslant m\geqslant n_0,\forall N\in\mathbb{N},
\norme{\sum_{k=0}^N (a^{(n)}_k-a_k)e_k}<\varepsilon
\]
On en d�duit les deux assertions qu'il restait � prouver.
\end{proof}

On va maintenant donner une nouvelle caract�risation aux suites basiques.

\begin{Def}
On dira qu'une suite $(e_n)_{n\in\mathbb{N}}$ de vecteurs non nuls admet une constante de base s'il existe un r�el $K>0$ tel que pour tout $p\leqslant q\in\mathbb{N}$ et $a_0,...,a_q\in\mathbb{K}$, on a
\[
\norme{\sum_{n=0}^p a_n e_n}\leqslant K\norme{\sum_{n=0}^q a_n e_n}
\]
La plus petite valeur qui v�rifie cette propri�t� est appel�e constante de base.\footnote{On dira \textit{une constante de base} pour une valeur qui v�rifie la propri�t�, et \textit{la constante de base} pour la plus petite valeur qui v�rifie la propri�t�.} \\
Une suite basique dont la constante de base vaut 1 est appel�e suite basique monotone.
\end{Def}

\begin{Rem}
Autrement dit, pour $p\leqslant q\in\mathbb{N}$, les projections
\[
P_{q, p}:\Span(e_0,...,e_q)\maps\Span(e_0,...,e_p)
\]
sont continues et de norme uniform�ment born�e.

Ceci implique que les projections $P_n:E\maps\Span(e_0,...,e_n)$ sont continues et de norme inf�rieure � $K$.
De plus, on se rend facilement compte en prenant $p=q$, qu'une constante de base ne peut �tre inf�rieure � 1.
\end{Rem}

\begin{Prop}\label{schauderbasique}
Dans un espace de Banach, une suite est basique si et seulement si elle admet une constante de base.
\end{Prop}
\begin{proof}
~

$\Rightarrow)$\\
%
On suppose que $(e_n)_{n\in\mathbb{N}}$ est une suite basique. L'espace d�crit par le lemme (\ref{espseriescgebanach}), que l'on notera $F$, est un espace de Banach. On note $E_0=\SpanAdh((e_n)_{n\in\mathbb{N}})$. L'application
\[
\begin{array}{lll}
F&\maps&E_0\\
(a_n)_{n\in\mathbb{N}}&\maps&\displaystyle\sum_{n\in\mathbb{N}}a_n e_n
\end{array}
\]
est bijective. La surjectivit� provient de la d�finition de $F$ et $E_0$, et l'injectivit� vient du fait que la suite $(e_n)_{n\in\mathbb{N}}$ est une suite basique. De plus, cette application est continue
\[
\norme{\sum_{n\in\mathbb{N}}a_n e_n}\leqslant \sup_{N\in\mathbb{N}}\norme{\sum_{n=0}^N a_n e_n}=\norme{(a_n)_{n\in\mathbb{N}}}_F
\]
D'o�, par le th�or�me des isomorphismes de Banach, elle est bicontinue. Il existe par cons�quent $K>0$ tel que pour toute suite $(a_n)_{n\in\mathbb{N}}$ de vecteurs de $F$,
\[
\sup_{N\in\mathbb{N}}\norme{\sum_{n=0}^N a_n e_n}
\leqslant
K\norme{\sum_{n\in\mathbb{N}}a_n e_n}
\]
Ce qui d�montre la premi�re implication.\newline

$\Leftarrow)$\\
%
Supposons que la suite admet $K>0$ comme constante de base. Soit $x\in\SpanAdh((e_n)_{n\in\mathbb{N}})$. On a une suite $(x^{(n)})_{n\in\mathbb{N}}$ de vecteurs de $\Span((e_n)_{n\in\mathbb{N}})$ qui converge vers $x$. Pour $n\in\mathbb{N}$, on r��crit
\[
x^{(n)}=\sum_{k=0}^{m_n}x^{(n)}_k e_k\text{, avec }m_n\in\mathbb{N}\text{, et pour }k\in\{0,...,m_n\}, x^{(n)}_k\in\mathbb{K}
\]
Cette suite �tant convergente, quelque soit $k\in\mathbb{N}$, la suite $\left(P_k(x^{(n)})\right)_{n\in\mathbb{N}}$ est aussi convergente, et converge vers $P_k(x)$.\\
Si $k=0$, $P_0(x^{(n)})=x^{(n)}_0 e_0\rightarrow x_0 e_0$ pour un scalaire $x_0$.\\
Si $k=1$, $P_1(x^{(n)})=x^{(n)}_0 e_0+x^{(n)}_1 e_1\rightarrow x_0 e_0+x_1 e_1$ pour un scalaire $x_1$.\\
...\\
Si $k=N$, $P_N(x^{(n)})=\sum\limits_{l=0}^N x^{(n)}_l e_l\rightarrow \sum\limits_{l=0}^N x_l e_l$ pour un scalaire $x_N$.\\
...

On obtient ainsi une suite $(x_n)_{n\in\mathbb{N}}$ telle que pour tout $n\in\mathbb{N}$,
\[
P_k(x)=\sum_{l=0}^k x_l e_l
\]
Montrons que $P_k(x)\rightarrow x$. Soit $\varepsilon>0$. Puisque $x^{(n)}\rightarrow x$, il existe $n\in\mathbb{N}$ tel que pour tout $m\geqslant n$, $\norme{x^{(m)}-x}<\varepsilon$. Prenons $k_0 = m_{n}$. Alors, quelque soit $k\geqslant k_0$, on a que $P_k(x^{(n)})=x^{(n)}$, d'o�
\[
\begin{array}{ll}
\norme{P_k(x)-x}
&\leqslant
\norme{P_k(x)-P_k(x^{(n)})}+
\norme{P_k(x^{(n)})-x^{(n)}}+
\norme{x^{(n)}-x}
\\&\leqslant
K\norme{x^{(n)}-x}+
\norme{x^{(n)}-x}
\leqslant
(1+K)\varepsilon
\end{array}
\]
Montrons que cette d�composition est unique. Prenons une deuxi�me suite de scalaires $(y_n)_{n\in\mathbb{N}}$ telle que $\sum\limits_{n\in\mathbb{N}}y_n e_n = x$. Alors, $\sum\limits_{n\in\mathbb{N}}(x_n-y_n)e_n = 0$, donc pour tout $N\in\mathbb{N}$,
\[
\sum_{n=0}^N(x_n-y_n)e_n
=
P_N\left(\sum_{n\in\mathbb{N}}(x_n-y_n)e_n\right)
=
P_N(0)
=
0
\]
On en d�duit alors, puisque les $(e_n)_{n\in\mathbb{N}}$ sont non nuls, que les d�compositions sont �gales, et donc qu'il n'y en a qu'une. Ceci termine la d�monstration.
\end{proof}

Soient $E$ un espace de Banach et $(e_n)_{n\in\mathbb{N}}$ une suite basique de $E$. Au regard de la propri�t� que l'on vient de d�montrer, on d�finit les applications lin�aires
\[
\begin{array}{lllll}
f_n&:&\SpanAdh((e_n)_{n\in\mathbb{N}})&\maps&\mathbb{K}\\
&:&\displaystyle x=\sum_{n\in\mathbb{N}}x_n e_n&\mapsto&x_n
\end{array}
\]
Ces applications sont continues, car
\[
\abs{f_0(x)}=\abs{x_0}=
\frac{\norme{x_0 e_0}}{\norme{e_0}}=
\frac{\norme{P_0(x)}}{\norme{e_0}}\leqslant
\frac{K\norme{x}}{\norme{e_0}}
\]
et
\[
\abs{f_{n+1}(x)}=\frac{\norme{x_{n+1} e_{n+1}}}{\norme{e_{n+1}}}
=
\frac{\norme{P_{n+1}(x)-P_n(x)}}{\norme{e_{n+1}}}
\leqslant
\frac{2K\norme{x}}{\norme{e_{n+1}}}
\]
Par cons�quent, elles peuvent �tre �tendues � $E$ par le th�or�me de Hahn-Banach (\ref{HBA}). Ces applications sont appel�es fonctionnelles coordonn�es si la suite $(e_n)_{n\in\mathbb{N}}$ est une base de Schauder.

Dans le cas o� $(e_n)_{n\in\mathbb{N}}$ est une base de Schauder, on obtient ce que l'on voulait, � savoir que pour une suite $(x_n)_{n\in\mathbb{N}}$ convergeant vers $x$, on a $f_k(x_n)\underset{n\rightarrow+\infty}{\longrightarrow} f_k(x)$ quelque soit $k\in\mathbb{N}$. On obtient de plus, pour un vecteur $x$, l'�criture suivante:
\[
x=\sum_{n\in\mathbb{N}}f_n(x)e_n.
\]

Pour la m�me raison, quelque soit $x\in\SpanAdh((e_n)_{n\in\mathbb{N}})$, on a que $P_n(x)\rightarrow x$. Autrement dit, la suite des projections $(P_n)_{n\in\mathbb{N}}$ converge pr�faiblement vers l'identit� sur $\SpanAdh((e_n)_{n\in\mathbb{N}})$.



On sait qu'un espace non s�parable ne contient aucune base de Schauder. D'ailleurs, �tre s�parable n'est pas suffisant pour forcer l'espace � contenir une base de Schauder. En effet, pour $p>1$, $p\neq 2$, l'espace $L^p([0,1])$ contient un sous-espace sans base de Schauder (voir \cite[p. 309-317]{PENFLO}). On va voir que pour les suites basiques, cela se passe diff�remment.

\begin{Lemme}\label{basiquexiste}
Soit $E$ un espace de Banach de dimension infinie contenant $(e_0,...,e_n)$ une suite finie de constante de base $K$. Quel que soit $\varepsilon>0$, on peut trouver $e_{n+1}\in S(E)$ tel que $(e_0,...,e_{n+1})$ a une constante de base inf�rieure � $K(1+\varepsilon)$.
\end{Lemme}
\begin{proof}
Soit $\varepsilon>0$, on pose $\delta=\frac{\varepsilon}{1+\varepsilon}$. Dans $\Span(e_0,...,e_n)$ la sph�re unit� est compacte, on a donc $(p_1,...,p_N)$ un $\delta$-r�seau de cette sph�re. Par le th�or�me de Hahn-Banach (\ref{HBF}), il existe $\varphi_k\in E\dual$ de norme 1 tel que $\varphi_k(p_k)=1$, quelque soit $k\in\{1,...,N\}$. Comme $E$ est de dimension infinie, il existe $e_{n+1}\in E$ tel que $\norme{e_{n+1}}=1$ et $\varphi_k(e_{n+1})=0$, quelque soit $k\in\{1,...,N\}$.

Soient $a_0,...,a_{n+1}\in\mathbb{K}$ tels que $\norme{\sum\limits_{k=0}^n a_k e_k}=1$. Il existe $m\in\{1,...,N\}$ v�rifiant
\[
\norme{p_m-\sum_{k=0}^n a_k e_k}
\leqslant \delta
\]
Alors
\begin{align*}
1 &\displaystyle
= \varphi_m\left(p_m+a_{n+1}e_{n+1}\right)
\leqslant
\norme{p_m+a_{n+1}e_{n+1}}
=
\norme{p_m+\sum_{k=0}^{n+1}a_k e_k-\sum_{k=0}^n a_k e_k}
\\&\displaystyle
\leqslant
\norme{p_m-\sum_{k=0}^n a_k e_k}+\norme{\sum_{k=0}^{n+1}a_k e_k}
\leqslant
\delta+\norme{\sum_{k=0}^{n+1}a_k e_k}
\end{align*}
D'o�
\[
\norme{\sum_{k=0}^{n+1}a_k e_k}
\geqslant
1-\delta
=
\frac{1}{1+\varepsilon}
\]
Par cons�quent
\[
(1+\varepsilon)\norme{\sum_{k=0}^{n+1}a_k e_k}
\geqslant
1 = \norme{\sum_{k=0}^{n}a_k e_k}
\]
Par homog�n��t�, le r�sultat s'�tend � toute suite finie de $n+1$ scalaires quelconques. Montrons maintenant que la suite finie $(e_0,...,e_{n+1})$ poss�de une constante de base inf�rieure � $K(1+\varepsilon)$. Les autres cas �tant �vidents, on consid�re uniquement le cas o� $p\leqslant n$ et $q=n+1$. Alors, pour $a_0,...,a_{n+1}\in\mathbb{K}$, on a
\[
\norme{\sum_{k=0}^p a_k e_k}
\leqslant
K\norme{\sum_{k=0}^n a_k e_k}
\leqslant
K(1+\varepsilon)\norme{\sum_{k=0}^{n+1} a_k e_k}
\]
\end{proof}
\begin{Corol}
Tout espace de Banach de dimension infinie admet une suite basique normalis�e. De plus, quelque soit $\varepsilon>0$, on peut choisir cette suite de telle sorte que sa constante de base soit inf�rieure � $1+\varepsilon$.
\end{Corol}
\begin{proof}
Soient $e_1\in S(E)$ et $\varepsilon>0$. Ce vecteur repr�sente une suite finie � un �l�ment de $S(E)$, de constante de base �gale � 1. On proc�de par induction sur le lemme (\ref{basiquexiste}), en choisissant $\varepsilon_n=\exp(\frac{1}{2^n}\frac{\ln(1+\varepsilon)}{2})-1>0$. � la $n^\text{i�me}$ �tape, la suite finie a une constante de base inf�rieure �

\[
\begin{array}{ll}
\displaystyle\prod_{k=0}^n(1+\varepsilon_k)
\hspace*{-0.3cm}\linebreak[0]
&\displaystyle=
\prod_{k=0}^n\exp\left(\frac{1}{2^k}\frac{\ln(1+\varepsilon)}{2}\right)
=
\exp\left(\sum_{k=0}^n\frac{1}{2^k}\frac{\ln(1+\varepsilon)}{2}\right)
\\&=
\exp\left(\left(2-\frac{1}{2^n}\right)\frac{\ln(1+\varepsilon)}{2}\right)
=
\exp\left(\ln(1+\varepsilon)-\frac{1}{2^n}\frac{\ln(1+\varepsilon)}{2}\right)
=
(1+\varepsilon)\exp\left(-\frac{1}{2^n}\frac{\ln(1+\varepsilon)}{2}\right)
\end{array}
\]

qui converge de mani�re croissante vers $1+\varepsilon$. Le lemme donne � chaque �tape un vecteur de norme 1, le corollaire est prouv�.
\end{proof}

\begin{Rem}
Avec le lemme (\ref{basiquexiste}), on ne peut pas faire mieux que le r�sultat que l'on vient de d�montrer. En effet, � partir du moment o� on a choisi le premier vecteur, la constante de base associ�e est �gale � 1. Ensuite, quelque soit $\varepsilon>0$ utilis�, le lemme ne garantit pas que la constante de base reste �gale � 1. L'existence d'une suite basique monotone est par cons�quent impossible � �tablir de cette mani�re.
\end{Rem}



\section{Base contractante, base compl�te, caract�risations du bidual}

\begin{Def}
Soit $E$ un espace de Banach contenant une base de Schauder $(e_n)_{n\in\mathbb{N}}$.\\
On dit que cette base est compl�te (boundedly-complete) si toute suite de scalaires $(a_n)_{n\in\mathbb{N}}$ telle que $\sup\limits_{N\in\mathbb{N}}\norme{\sum\limits_{n=0}^N a_n e_n}<+\infty$ rend la s�rie $\sum\limits_{n\in\mathbb{N}} a_n e_n$ convergente.\\
On dit que cette base est contractante (shrinking) si quelque soit $x\dual\in E\dual$,
\[
\sup\left\{
\abs{x\dual(x)}
\left|
\norme{x}=1\text{ et }
x=\sum_{j\geqslant n}x_j e_j,\text{ pour }(a_j)_{j\geqslant n}\in\mathbb{K}^\mathbb{N}
\right.
\right\}
\underset{n\rightarrow+\infty}{\longrightarrow}0.
\]
\end{Def}
\begin{Rem}
La d�finition d'une base contractante peut s'�crire de mani�re plus concise:
\[
\forall x\dual\in E\dual,
\sup\left\{
\abs{x\dual(x)}
\text{ tel que }
\norme{x}=1\text{ et }P_n(x)=0
\right\}
\underset{n\rightarrow+\infty}{\longrightarrow}0.
\]
\end{Rem}

\subsection*{Exemples}
\subsubsection*{Base de Schauder contractante, non compl�te}
On prend la base canonique dans $c_0$. Elle est contractante, et n'est pas compl�te. Pour d�montrer la premi�re assertion, on proc�de par l'absurde.

Supposons que la base ne soit pas contractante. Il existe alors $x\dual\in l^1$ et $\varepsilon>0$ tels que pour tout $n\in\mathbb{N}$, il existe $N\geqslant n$ tel que
\[
\sup\left\{
\abs{x\dual(x)}
\left|
\norme{x}=1\text{ et }
x=\sum_{j\geqslant N}x_j e_j,\text{ pour }(a_j)_{j\geqslant n}\in\mathbb{K}^\mathbb{N}
\right.
\right\}
>\varepsilon
\]
Puisque $x\dual\in l^1$, $x\dual=(x\dual_n)_{n\in\mathbb{N}}$ et il existe $n_0\in\mathbb{N}$ tel que $\sum\limits_{n\geqslant n_0}\abs{x\dual_n}<\varepsilon$.

Puisque la base n'est pas contractante, il existe $N\geqslant n_0$ et $x\in S(c_0)$ tel que $x=\sum\limits_{n\geqslant N}x_n e_n$ et $\abs{x\dual(x)}>\varepsilon$. Cependant,
\[
\abs{x\dual(x)}
=
\abs{\sum_{n\geqslant N}x\dual_n x_n}
\leqslant
\sum_{n\geqslant N}\abs{x\dual_n}
\leqslant
\varepsilon
\]
Ce qui est absurde.

Montrons qu'elle n'est pas compl�te. Prenons comme suite de scalaires la suite constante 1. Alors, pour tout $N\in\mathbb{N}$,
$\sum\limits_{n=0}^N e_n\in S(c_0)$. Pourtant, la suite $\left(\sum\limits_{n=0}^N e_n\right)_{N\in\mathbb{N}}$ n'est pas de Cauchy, donc la base n'est pas compl�te.

\subsubsection*{Base de Schauder compl�te, non contractante}
On prend la base canonique dans $l^1$.

Soit $(a_n)_{n\in\mathbb{N}}$ une suite de scalaires telle que $\sup\limits_{N\in\mathbb{N}}\norme{\sum\limits_{n=0}^N a_n e_n}<+\infty$. On a que
\[
\sup_{N\in\mathbb{N}}\norme{\sum_{n=0}^N a_n e_n}
=
\sup_{N\in\mathbb{N}}\sum_{n=0}^N\abs{a_n}
=
\sum_{n\in\mathbb{N}}\abs{a_n}<+\infty
\]
Dans ce cas, la suite $(a_n)_{n\in\mathbb{N}}$ est un �l�ment de $l^1$, et la suite des sommes partielles converge bien vers cet �l�ment, puisque la base canonique est une base de Schauder de $l^1$.

Soit $x\dual=(1,...,1,...)\in l^\infty$. Quelque soit $n\in\mathbb{N}_0$, le vecteur $e_n$ est de norme 1 et a une $(n-1)^\text{i�me}$ projection qui est nulle. Puisque $x\dual(e_n)=1$, on en d�duit que la base n'est pas contractante.

\subsubsection*{Base de Schauder non contractante, non compl�te}
Dans l'espace $c_0$, on prend la base $(s_n)_{n\in\mathbb{N}} = \left(\sum\limits_{k=0}^n e_k\right)_{n\in\mathbb{N}}$.

D'une part elle n'est pas contractante car $f(s_n)=1$ quelque soit $n\in\mathbb{N}$, o� $f:(c_0\maps\mathbb{K}:x\mapsto x_0)$ est la premi�re fonctionnelle coordonn�e associ�e � la base $(e_n)_{n\in\mathbb{N}}$.

D'autre part elle n'est pas compl�te car en prenant pour suite de scalaires la suite $((-1)^n)_{n\in\mathbb{N}}$, on construit une suite d'�l�ments de norme 1, qui n'est pas de Cauchy.\newline

Nous ne donnons pas d'exemple de base de Schauder contractante et compl�te, car l'existence d'une telle base fera l'objet d'un th�or�me.

Ceci fait, donnons quelques caract�risations de ces deux nouveaux types de base. On verra, au travers celles-ci, que les nouveaux types de bases consid�r�s sont tr�s puissants.

\begin{Prop}\label{contractantessifonctionnelleschauder}
Dans un espace de Banach, si une base de Schauder est contractante, alors la suite de fonctionnelles coordonn�es associ�e est une base de Schauder, et invers�ment.
\end{Prop}
\begin{proof}
Soient $(e_n)_{n\in\mathbb{N}}$ une base de Schauder de $E$, et $(f_n)_{n\in\mathbb{N}}$ les fonctionnelles coordonn�es qui lui sont associ�es. Posons $K$ la constante de base de $(e_n)_{n\in\mathbb{N}}$.

Supposons que $(e_n)_{n\in\mathbb{N}}$ soit contractante. Soient $x\dual\in E\dual$ et $x=\sum\limits_{n\in\mathbb{N}}x_n e_n\in E$. On a
\[
x\dual(x)
=
x\dual\left(\sum_{n\in\mathbb{N}}x_n e_n\right)
=
\sum_{n\in\mathbb{N}}x_n x\dual(e_n)
=
\sum_{n\in\mathbb{N}}f_n(x) x\dual(e_n)
\]
Une d�composition raisonnable pour $x\dual$ est alors $\sum\limits_{n\in\mathbb{N}}x\dual(e_n)f_n$. Montrons que cette s�rie est convergente (et donc converge bien vers $x\dual$).

Soit $\varepsilon>0$. La base $(e_n)_{n\in\mathbb{N}}$ �tant contractante, il existe $n_0\in\mathbb{N}$ tel que pour tout $n\geqslant n_0$, on a
\[
\sup\left\{
\abs{x\dual(x)}
\text{ tel que }
\norme{x}=1\text{ et }P_n(x)=0
\right\}<\varepsilon
\]
Soient $n>m\geqslant n_0$, et $\norme{x}=1$. Si $\sum\limits_{k=m+1}^n x\dual(e_k)f_k(x)=0$, alors $\abs{x\dual(x)}<\varepsilon$. Si � l'inverse $\sum\limits_{k=m+1}^n x\dual(e_k)f_k(x)\neq 0$, on a que
\[
\begin{array}{l}
\displaystyle
\abs{\sum_{k=m+1}^n x\dual(e_k)f_k(x)}
=
\abs{x\dual\left(\sum_{k=m+1}^n x_k e_k\right)}
=
\abs{x\dual\left(P_n(x)-P_m(x)\right)}
\\ \displaystyle
=
\abs{x\dual\left(\frac{P_n(x)-P_m(x)}{\norme{P_n(x)-P_m(x)}}\right)}\norme{P_n(x)-P_m(x)}
\leqslant
2K\abs{x\dual\left(\frac{P_n(x)-P_m(x)}{\norme{P_n(x)-P_m(x)}}\right)}
\leqslant
2K\varepsilon
\end{array}
\]
Par passage au supremum sur $x$, on obtient que $\norme{\sum\limits_{k=m+1}^n x\dual(e_k)f_k}<2K\varepsilon$, la s�rie est donc convergente. Montrons qu'il n'y a pas d'autre d�composition que celle que l'on vient d'�crire. Soit une deuxi�me d�composition $\sum\limits_{n\in\mathbb{N}}y_n f_n$ de $x\dual$. Soit $n\in\mathbb{N}$, on a
\[
0=\left(\sum_{k\in\mathbb{N}}(y_k-x\dual(e_k))f_k\right)(e_n)=y_n-x\dual(e_n)
\]

Maintenant, supposons que la suite de fonctionnelles coordonn�es soit une base de Schauder. Soient $x\dual\in E\dual$ et $\varepsilon>0$. Alors, $\sum\limits_{n\in\mathbb{N}}x\dual(e_n)f_n=x\dual$, il existe $n_0\in\mathbb{N}_0$ tel que pour tout $n\geqslant n_0$, $\norme{\sum\limits_{j\geqslant n}x\dual(e_j)f_j}<\varepsilon$. Soient $n\geqslant n_0$ et $x\in E$ tel que $\norme{x}=1$ et $P_n(x)=0$, on a que
\[
\abs{x\dual(x)}
=
\abs{x\dual\left(\sum_{j\in\mathbb{N}}x_j e_j\right)}
=
\abs{\sum_{j\in\mathbb{N}}x\dual(e_j)f_j(x)}
=
\abs{\sum_{j\geqslant n}x\dual(e_j)f_j(x)}
\leqslant
\norme{\sum_{j\geqslant n}x\dual(e_j)f_j}
<
\varepsilon
\]
Par passage au supremum sur $x$, on d�duit que la base $(e_n)_{n\in\mathbb{N}}$ est contractante.
\end{proof}
\begin{Rem}
Dans la d�monstration, on a pu voir que si la suite de fonctionnelles coordonn�es $(f_n)_{n\in\mathbb{N}}$ est une base de Schauder, alors toute fonctionnelle $x\dual$ s'�crit
\[
x\dual=\sum_{n\in\mathbb{N}}x\dual(e_n)f_n
\]
\end{Rem}

\begin{Rem}
On garde les notations de la propri�t�, et on suppose que $(e_n)_{n\in\mathbb{N}}$ est contractante. Soient $n\in\mathbb{N}$, $x\in E$ et $x\dual\in E\dual$. On a
\[
\adj{P_n}(x\dual)(x)
=
x\dual(P_n(x))
=
x\dual\left(\sum_{k=0}^n x_k e_k\right)
=
\sum_{k=0}^n x_k x\dual(e_k)
=
\sum_{k=0}^n f_k(x) x\dual(e_k)
\]
Si on note $P_n\dual$ la $n^\text{i�me}$ projection associ�e � $(f_n)_{n\in\mathbb{N}}$, alors on a que $\adj{P_n}=P_n\dual$. On en d�duit, puisque $\norme{\adj{P_n}}=\norme{P_n}$, que les bases $(e_n)_{n\in\mathbb{N}}$ et $(f_n)_{n\in\mathbb{N}}$ ont la m�me constante de base.
\end{Rem}

\begin{Rem}\end{Rem}
Toujours sous les m�mes hypoth�ses, on v�rifie ais�ment que les fonctionnelles coordonn�es associ�es � $(f_n)_{n\in\mathbb{N}}$ sont �gales � $(ev_{e_n})_{n\in\mathbb{N}}$. De plus, si la base $(f_n)_{n\in\mathbb{N}}$ est contractante, alors l'espace est r�flexif.


\section{Caract�risations du bidual}

\begin{Thm}\label{dualisometriqueschauder}
Soit $E$ un espace de Banach contenant $(e_n)_{n\in\mathbb{N}}$ une base de Schauder contractante. Alors, on peut identifier $E\bidual$ isomorphiquement � l'espace
\[
F=
\left\{
(a_n)_{n\in\mathbb{N}}\in\mathbb{K}~\left|~\sup_{N\in\mathbb{N}}\norme{\sum_{n=0}^N a_n e_n}<+\infty\right.
\right\}
\]
muni de la norme
\[
\norme{(a_n)_{n\in\mathbb{N}}}
=
\sup_{N\in\mathbb{N}}\norme{\sum_{n=0}^N a_n e_n}
\]
via la correspondance $x\bidual\longleftrightarrow(x\bidual(f_n))_{n\in\mathbb{N}}$, o� les $(f_n)_{n\in\mathbb{N}}$ sont les fonctionnelles coordonn�es. Si de plus la base choisie est monotone, alors la correspondance d�crit une isom�trie.
\end{Thm}
\begin{proof}
Par la proposition (\ref{contractantessifonctionnelleschauder}), on sait que les fonctionnelles coordonn�es forment une base de Schauder de $E\dual$. Notons $P_n\dual$ les projections associ�es aux $(f_n)_{n\in\mathbb{N}}$.

Nommons $F$ l'espace que l'on veut identifier � $E\bidual$. On va d�montrer que l'application lin�aire
\[
\begin{array}{lllll}
T&:&E\bidual&\maps&F\\
&:&x\bidual&\mapsto&(x\bidual(f_n))_{n\in\mathbb{N}}
\end{array}
\]
est une bijection bicontinue. Pour cela, commen�ons par v�rifier que cette application est bien d�finie, c'est-�-dire que quelque soit $x\bidual$, on a $\norme{T(x\bidual)}<+\infty$. On passera par les adjointes des projections.

Soient $x\dual\in E\dual$, $x\bidual\in E\bidual$ et $n\in\mathbb{N}$. On a
\[
\biadj{P}_n(x\bidual)(x\dual)
=
x\bidual(P_n\dual(x\dual))
=
\sum_{k=0}^n x\dual(e_k)x\bidual(f_k)
\]
On en d�duit que $\biadj{P}_n(x\bidual)=\sum\limits_{k=0}^n x\bidual(f_k) \: ev_{e_k}$. Cependant, on a que
\[
\norme{\biadj{P}_n(x\bidual)}
\leqslant
\norme{\biadj{P}_n}\norme{x\bidual}
=
\norme{P_n}\norme{x\bidual}
\leqslant
K\norme{x\bidual}
\]
Donc, par passage au supremum sur $n$, on obtient que
\[
\sup_{n\in\mathbb{N}}\norme{\sum_{k=0}^n x\bidual(f_k) \: ev_{e_k}}
=
\norme{T(x\bidual)}
\leqslant
K\norme{x\bidual}
\]
On vient de d�montrer que $T$ est bien d�finie, et qu'elle est continue.

On montre que l'application est injective. Soit $x\bidual\in E\bidual$ tel que pour tout $n\in\mathbb{N}$, on a $x\bidual(f_n)=0$. Puisque la base $(e_n)_{n\in\mathbb{N}}$ est contractante, la suite des fonctionnelles coordonn�es $(f_n)_{n\in\mathbb{N}}$ est une base de Schauder. Alors, pour tout $x\dual\in E\dual$ on a

\[
x\bidual(x\dual)
=
x\bidual\left(\sum_{n\in\mathbb{N}}x\dual(e_n)f_n\right)
=
\sum_{n\in\mathbb{N}}x\dual(e_n)x\bidual(f_n)
=
0
\]

Ce qui implique que $x\bidual=0$.

Montrons qu'elle est surjective. Soient $a\in F$, $x\dual\in E\dual$ et $\varepsilon>0$. La base $(e_n)_{n\in\mathbb{N}}$ �tant contractante, on a $n_0\in\mathbb{N}$ tel que si $n\geqslant n_0$, alors
\[
\sup\left\{
\abs{x\dual(x)}
\left|
\norme{x}=1\text{ et }
x=\sum_{j\geqslant n}x_j e_j,\text{ pour }(a_j)_{j\geqslant n}\in\mathbb{K}^\mathbb{N}
\right.
\right\}
<\varepsilon
\]
Soient $n>m\geqslant n_0$. Si $\sum\limits_{k=m}^n a_k e_k=0$, son image par $x\dual$ est nulle. Dans l'autre cas, puisque la suite $\left(\sum\limits_{k=0}^n a_k e_k\right)_{n\in\mathbb{N}}$ est born�e, on a
\[
\begin{array}{ll}
\displaystyle
\abs{\left(\sum_{k=m}^n a_k ev_{e_k}\right)(x\dual)}
&=\displaystyle
\abs{x\dual\left(\sum_{k=m}^n a_k e_k\right)}
=
\abs{x\dual\left(\frac{\sum\limits_{k=m}^n a_k e_k}{\norme{\sum\limits_{k=m}^n a_k e_k}}\right)}\norme{\sum_{k=m}^n a_k e_k}
<
\varepsilon\norme{\sum_{k=m}^n a_k e_k}
\\&\displaystyle
\leqslant
2K\varepsilon\norme{\sum_{k=0}^n a_k e_k}
\leqslant
2K\varepsilon\sup_{N\in\mathbb{N}}\norme{\sum_{k=0}^N a_k e_k}
\end{array}
\]

Ce qui signifie que la s�rie $\displaystyle\sum_{n\in\mathbb{N}}a_n x\dual(e_n)$ converge. On peut alors d�finir l'application lin�aire

\[
\begin{array}{lllll}
x\bidual&:&E\dual&\maps&\mathbb{K}\\
&:&x\dual&\mapsto&\displaystyle\sum_{n\in\mathbb{N}}a_n x\dual(e_n)
\end{array}
\]

Cette application est continue, car si $x\in B(E\dual)$, on a

\[
\abs{x\bidual(x\dual)}
=
\lim_{n\in\mathbb{N}}\abs{\sum_{n\in\mathbb{N}}a_n x\dual(e_n)}
\leqslant
\lim_{n\in\mathbb{N}}\norme{x\dual}\norme{\sum_{n\in\mathbb{N}}a_n e_n}
\leqslant
\norme{a}
\]

� ce stade, il est �vident que $T(x\bidual)=a$, l'application $T$ est donc surjective.

Enfin, finissons par montrer que l'inverse est continue de norme inf�rieure � 1. Ceci confirmera que $T$ est bien un isomorphisme, et dans le cas o� $K=1$ que c'est m�me une isom�trie.

Soit $x\bidual\in E\bidual$, et soit $(x_n\dual)_{n\in\mathbb{N}}$ une suite � valeur dans $S(E\dual)$ telle que pour tout $n\in\mathbb{N}_0$, $\abs{x\bidual(x_n\dual)-\norme{x\bidual}}\leqslant\frac{1}{n}$. Puisque $(f_n)_{n\in\mathbb{N}}$ est une base de Schauder, quelque soit $n\in\mathbb{N}$ on a $P_k\dual(x_n\dual)\underset{k\rightarrow+\infty}{\longrightarrow}x_n\dual$. Puisque $\norme{x_n\dual}=1$, on a �galement que $\frac{P_k\dual(x_n\dual)}{\norme{P_k\dual(x_n\dual)}}\underset{k\rightarrow+\infty}{\longrightarrow}x_n\dual$. Soit alors une suite croissante $(k_n)_{n\in\mathbb{N}}$ telle que pour tout $n\in\mathbb{N}$, on a
\[
\norme{\frac{P_{k_n}\dual(x_n\dual)}{\norme{P_{k_n}\dual(x_n\dual)}}-x_n\dual}\leqslant\frac{1}{n}
\]
Soit $n\in\mathbb{N}$. On a
\[
\abs{x\bidual\left(\frac{P_{k_n}\dual(x_n\dual)}{\norme{P_{k_n}\dual(x_n\dual)}}\right)-\norme{x\bidual}}
\leqslant
\abs{x\bidual\left(\frac{P_{k_n}\dual(x_n\dual)}{\norme{P_{k_n}\dual(x_n\dual)}}-x_n\dual\right)}
+ \abs{x\bidual(x_n\dual)-\norme{x\bidual}}
\leqslant \frac{\norme{x\bidual}}{n} + \frac{1}{n}
\]
D'o� $\abs{x\bidual\left(\frac{P_{k_n}\dual(x_n\dual)}{\norme{P_{k_n}\dual(x_n\dual)}}\right)}\rightarrow\norme{x\bidual}$. De plus, on a, pour tout $n\in\mathbb{N}$, que
\[
\begin{array}{ll}
\displaystyle
\norme{\sum_{l=0}^{k_n} x\bidual(f_l) \: ev_{e_l}}
&\geqslant\displaystyle
\abs{\left(\sum_{l=0}^{k_n} x\bidual(f_l) \: ev_{e_l}\right)\left(\frac{P_{k_n}\dual(x_n\dual)}{\norme{P_{k_n}\dual(x_n\dual)}}\right)}
=
\abs{\sum_{l=0}^{k_n} x\bidual(f_l)\frac{x_n\dual(e_l)f_l(e_l)}{\norme{P_{k_n}\dual(x_n\dual)}}}
\\&\displaystyle
=
\abs{\sum_{l=0}^{k_n} x\bidual(f_l)\frac{x_n\dual(e_l)}{\norme{P_{k_n}\dual(x_n\dual)}}}
=
\abs{x\bidual\left(\frac{\sum\limits_{l=0}^{k_n}x_n\dual(e_l)f_l}{\norme{P_{k_n}\dual(x_n\dual)}}\right)}
=
\abs{x\bidual\left(\frac{P_{k_n}\dual(x_n\dual)}{\norme{P_{k_n}\dual(x_n\dual)}}\right)}
\end{array}
\]
Par passage au supremum sur $n$, on obtient l'in�galit� souhait�e.
\end{proof}

On a ainsi une caract�risation compl�te du bidual en fonction de la base de Schauder monotone que l'on s'est fix�e. On peut �galement remarquer que cette caract�risation ne n�cessite aucune connaissance du dual. Ceci est tr�s important pour �tudier certains espaces de Banach comme par exemple l'espace de James. Mais on va voir que les liens entre les bases de Schauder et le bidual ne s'arr�tent pas l�.

\begin{Corol}Si un espace de Banach contient une base de Schauder monotone, contractante et compl�te, alors il est r�flexif.\end{Corol}

Bien que celle-ci soit simple, nous ne donnerons pas de d�monstration de ce corollaire, car on va d�montrer un r�sultat plus puissant qui ne n�cessite pas la monotonicit� de la base choisie.

\begin{Thm}[James]\label{JamesSchauder}
Dans un espace r�flexif, toute base de Schauder est compl�te et contractante. Inversement, si un espace de Banach admet une base de Schauder compl�te et contractante, alors il est r�flexif.
\end{Thm}
\begin{proof}
~

$\Rightarrow)$\\
%
Supposons que $E$ soit un espace r�flexif qui contient une base de Schauder $(e_n)_{n\in\mathbb{N}}$. Montrons qu'elle est compl�te et contractante.

Soit $(a_n)_{n\in\mathbb{N}}$ une suite de scalaires telle que $\sup\limits_{N\in\mathbb{N}}\norme{\sum\limits_{n=0}^N a_n e_n}<+\infty$.\\
Par le th�or�me de Kakutani (\ref{kakutani}), comme l'espace est r�flexif, la boule unit� est faiblement compacte. Donc par le th�or�me d'Eberlein-\v{S}mulian (\ref{eberlein}), elle est faiblement s�quentiellement compacte. Alors, il existe une suite $(n_k)_{k\in\mathbb{N}}$ telle que $\left(\sum\limits_{n=0}^{n_k} a_n e_n\right)_{k\in\mathbb{N}}$ converge faiblement vers un �l�ment $a\in E$.\\
Soit $n\in\mathbb{N}$. On a que $f_n\left(\sum\limits_{l=0}^{n_k} a_l e_l\right)\underset{k\rightarrow+\infty}{\longrightarrow} a_n$, donc par unicit� de la limite, $f_n(a)=a_n$, et puisque $a=\sum\limits_{n\in\mathbb{N}}f_n(a)e_n=\sum\limits_{n\in\mathbb{N}}a_n e_n$, la base est compl�te.

Supposons qu'elle ne soit pas contractante. Alors, il existe $x\dual\in E\dual$, $\varepsilon>0$, une suite croissante $(n_k)_{n\in\mathbb{N}}$ de naturels et une suite $(v_k)_{k\in\mathbb{N}}$ tous de norme 1, tels que $P_{n_k}(v_k)=0$ et $\abs{x\dual(v_k)}>\varepsilon$ pour tout $k\in\mathbb{N}$. Alors, la suite $(v_k)_{k\in\mathbb{N}}$ �tant born�e et la boule unit� �tant faiblement compacte, le th�or�me d'Eberlein-\v{S}mulian nous dit qu'elle admet une sous-suite faiblement convergente. De plus, quelque soit $n\in\mathbb{N}$, on a que $f_n(v_k)\rightarrow 0$, donc $(v_k)_{k\in\mathbb{N}}$ converge faiblement vers 0. En particulier, on a que $x\dual(v_k)\rightarrow 0$, ce qui am�ne � une contradiction.\newline

$\Leftarrow)$\\
%
R�ciproquement, supposons que l'on ait un espace de Banach $E$ qui contient une base de Schauder $(e_n)_{n\in\mathbb{N}}$ compl�te et contractante.

Soient $x\bidual\in E\bidual$ et $n\in\mathbb{N}$. On a
\[
\norme{\sum_{k=0}^n x\bidual(f_k)e_k}
=
\norme{\sum_{k=0}^n x\bidual(f_k)ev_{e_k}}
=
\norme{\biadj{P_n}(x\bidual)}
\leqslant
\norme{\biadj{P_n}}\norme{(x\bidual)}
=
\norme{P_n}\norme{(x\bidual)}
\leqslant
K\norme{x\bidual}
\]
D'o� $\sup\limits_{n\in\mathbb{N}}\norme{\sum\limits_{k=0}^n x\bidual(f_k)e_k}<+\infty$.

Puisque la base $(e_n)_{n\in\mathbb{N}}$ est compl�te, il existe $x\in E$ tel que $x=\sum\limits_{n\in\mathbb{N}}x\bidual(f_n)e_n$. On va montrer que $ev_x=x\bidual$.

Soit $x\dual\in E\dual$. Par la proposition (\ref{contractantessifonctionnelleschauder}), comme $(e_n)_{n\in\mathbb{N}}$ est contractante, les fonctionnelles coordonn�es forment une base de Schauder de $E\dual$, et $x\dual=\sum\limits_{n\in\mathbb{N}}x\dual(e_n)f_n$. D'une part, on a que
\[
x\bidual(x\dual)
=
x\bidual\left(\sum_{n\in\mathbb{N}}x\dual(e_n)f_n\right)
=
\sum_{n\in\mathbb{N}}x\dual(e_n)x\bidual(f_n)
\]
D'autre part, on a �galement que

\[
\begin{array}{ll}
ev_x(x\dual)
&
\displaystyle
=
\left(\sum_{n\in\mathbb{N}}x\dual(e_n)f_n\right)\left(\sum_{k\in\mathbb{N}}x\bidual(f_k)e_k\right)
=
\sum_{n\in\mathbb{N}}\left(x\dual(e_n)\sum_{k\in\mathbb{N}}x\bidual(f_k)f_n(e_k)\right)
\\
&
\displaystyle
=
\sum_{n\in\mathbb{N}}x\dual(e_n)x\bidual(f_n)
\end{array}
\]

Donc, quelque soit $x\dual\in E\dual$, on a $ev_x(x\dual)=x\bidual(x\dual)$, par cons�quent $ev_x=x\bidual$, et $E$ est r�flexif.
\end{proof}

Ce th�or�me est tr�s important, car il permet de savoir si un espace est r�flexif en utilisant une seule base de Schauder. En effet, si on en a une qui est compl�te et contractante, l'espace est r�flexif. En revanche, si l'une des deux conditions n'est pas v�rifi�e, l'espace n'est pas r�flexif (sinon, toutes ses bases de Schauder, en particulier celle fix�e, seraient compl�tes et contractantes).

Une deuxi�me application du th�or�me se situe dans un espace r�flexif. On sait que l'injection canonique �tant une isom�trie bijective, toute base de Schauder s'identifie � une base de Schauder du bidual. Puisque l'espace est r�flexif, ses bases de Schauder sont automatiquement contractantes, ce qui implique que la suite des fonctionnelles coordonn�es soit, elle aussi, une base de Schauder. L'int�r�t est qu'ici, on peut en d�duire une base de Schauder du dual sans faire le moindre effort, ce qui n'�tait pas possible avant.

