\begin{Prop}\label{equivreflexif}
Les assertions suivantes sont �quivalentes:
\begin{enumerate}[(1)]
\item $E$ est r�flexif
\item $E\dual$ est r�flexif
\item tout sous-espace vectoriel ferm� de $E$ est r�flexif
\end{enumerate}
\end{Prop}
\begin{proof}
(1) $\Rightarrow$ (2)\\
Notons $i$ l'injection canonique de $E$, et $j$ l'injection canonique de $E\dual$. On veut montrer que $j$ est surjectif. Puisqu'on a l'injectivit�, cela revient � d�montrer que $j$ est bijective, et donc de mani�re �quivalente qu'il admet un inverse. Pour ce faire, on va montrer que $\adj{i}$ est l'inverse de $j$. Pour cela, il suffira de v�rifier que \hbox{$j\circ\adj{i}=Id_{E\tridual}$}.

Soient $x\tridual\in E\tridual$ et $x\bidual\in E\bidual$. Puisque $E$ est r�flexif, il existe $x\in E$ tel que $i(x)=x\bidual$. On calcule
\[
j(\adj{i}(x\tridual))(x\bidual)=
x\bidual(\adj{i}(x\tridual))=
ev_x(\adj{i}(x\tridual))=
\adj{i}(x\tridual)(x)=
x\tridual(i(x))=
x\tridual(x\bidual)
\]
Qui se traduit par
\[
\begin{array}{l}
\forall x\tridual\in E\tridual, \forall x\bidual\in E\bidual, j(\adj{i}(x\tridual))(x\bidual)=x\tridual(x\bidual)\\
\Rightarrow \forall x\tridual\in E\tridual, j(\adj{i}(x\tridual))=x\tridual\\
\phantom{\Rightarrow} \Rightarrow j\circ\adj{i}=Id_{E\tridual}\\
\end{array}
\]

Par cons�quent, $\adj{i}$ est bien l'inverse de $j$, donc $j$ est surjectif et $E\dual$ r�flexif.\newline

(1) $\Rightarrow$ (3)\\
Puisque $F$ est ferm�, l'application lin�aire
\[
\begin{array}{lllll}
\pi&:&E\dual&\maps&F\dual\\
&:&x\dual&\mapsto&\restrictions{x\dual}{F}
\end{array}
\]
est bien d�finie et continue. Le th�or�me de Hahn-Banach (\ref{HBA}) dit exactement que cette application est surjective:

Soit $y\bidual\in F\bidual$. Alors $y\bidual\circ\pi\in E\bidual$. Comme $E$ est r�flexif, il existe $x\in E$ tel que $ev_x=y\bidual\circ\pi$. Pour montrer que $x\in F$, on proc�de par l'absurde. Dans ce cas, le th�or�me de Hahn-Banach (\ref{HBA}) affirme qu'il existe une fonctionnelle continue $x\dual\in E\dual$ qui s'annule sur $F$ mais pas en $x$. Alors, on a que
\[
ev_x(x\dual)
=
y\bidual(\pi(x\dual))
=
0
\]
Ce qui est absurde. Par cons�quent, $x\in F$. Alors quelque soit $x\dual\in E\dual$, on a
\[
y\bidual(\pi(x\dual))
=
(y\bidual\circ\pi)(x\dual)
=
x\dual(x)
=
(\pi(x\dual))(x)
\]

Puisque $\pi$ est surjective, cela revient � dire que $y\bidual(y\dual)=y\dual(x)$, et ce quelque soit $y\dual\in F\dual$. Donc, $ev_x=y\bidual$.\newline

(2) $\Rightarrow$ (1)\newline

On sait que $E\dual$ est r�flexif, donc $E\bidual$ aussi, donc tout sous-espace ferm� de $E\bidual$ aussi. En particulier, c'est valable pour $i(E)$ qui est isomorphe � $E$, donc $E$ est r�flexif.

(3) $\Rightarrow$ (1) est �vident.
\end{proof}

\begin{Prop}
Soient $E$ un espace r�flexif et $F$ un sous-espace vectoriel ferm� de $E$. Alors, l'espace quotient $E/F$ est r�flexif.
\end{Prop}
\begin{proof}
Par la proposition (\ref{equivreflexif}), tout sous-espace ferm� de $E\dual$ est r�flexif, en particulier $F^\perp$ (voir (\ref{espaceortho})). Par la proposition (\ref{ForthoisoEsurF}), le dual de $E/F$ est isom�trique � $F^\perp$, et est r�flexif par (\ref{isoreflex}). On conclut avec (\ref{equivreflexif}).
\end{proof}

