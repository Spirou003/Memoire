\begin{Lemme}\label{constantebaseincondnorme}
Soit une suite $(e_n)_{n\in\mathbb{N}}$ de $E$, supposons qu'elle admet une constante de base inconditionnelle $K>0$. On d�finit sur $\SpanAdh((e_n)_{n\in\mathbb{N}})$ la norme

\[
\triplenorme{x}
=
\sup\left\{
\norme{\sum_{n\in F} \lambda_n x_n e_n} \text{tel que } \emptyset\neq F\subseteq\mathbb{N}\text{ fini}, \abs{\lambda_n}\leqslant 1,\forall n\in F
\right\}
\]

Alors

\begin{enumerate}[(1)]
\item Pour tout $x\in\SpanAdh((e_n)_{n\in\mathbb{N}})$, $\norme{x}\leqslant\triplenorme{x}\leqslant 4K\norme{x}$
\item $(e_n)_{n\in\mathbb{N}}$ a une constante de base inconditionnelle qui vaut 1 pour $(\SpanAdh((e_n)_{n\in\mathbb{N}}),\triplenorme{.})$
\item Pour toute paire d'ensembles finis disjoints non vides de naturels $A$ et $B$, pour toute suite de scalaires $(z_n)_{n\in\mathbb{N}}$, si on note

\[
x=\sum_{n\in A} z_n e_n ~~~~~~ y=\sum_{n\in B} z_n e_n
\]

alors la fonction

\[
\begin{array}{lll}
\mathbb{K}&\maps&\mathbb{R}^{+}\\
\lambda&\mapsto&\triplenorme{x+\lambda y}
\end{array}
\]

est convexe, invariante en $\Arg(\lambda)$ et croissante en $\abs{\lambda}$
\end{enumerate}
\end{Lemme}

\begin{proof}
~

(1)\\
%
Soit $x\in\SpanAdh((e_n)_{n\in\mathbb{N}})$. En prenant $F=\{1,...,n\}$, pour tout $k\in F$, $\lambda_k=1$, et en faisant tendre $n$ vers l'infini, on obtient la premi�re in�galit�. R�ciproquement, soient $x\in\SpanAdh((e_n)_{n\in\mathbb{N}})$, $F\subseteq\mathbb{N}$ un ensemble fini non vide et $\abs{\lambda_n}\leqslant 1$ pour tout $n\in F$. On a

\[
\norme{\sum_{n\in F} \lambda_n x_n e_n}
\leqslant
\norme{\sum_{n\in F} \Re(\lambda_n) x_n e_n}
+
\norme{\sum_{n\in F} \Im(\lambda_n) x_n e_n}
\]

Les deux sommes seront tra�t�es identiquement. Alors, on suppose pour simplifier les notations que $\lambda_n\in[-1,1]$, quelque soit $n\in F$. On consid�re $E_\mathbb{R}$ l'�quivalent r�el de $E$. Par le th�or�me de Hahn-Banach (\ref{HBF}) il existe $x\dual_\mathbb{R}\in S(E_\mathbb{R}\dual)$ tel que

\[
\norme{\sum_{n\in F}\lambda_n x_n e_n}
=
x\dual_\mathbb{R}\left(\sum_{n\in F}\lambda_n x_n e_n\right)
\]

Alors

\[
\norme{\sum_{n\in F} \lambda_n x_n e_n}
\leqslant
\sum_{n\in F}\abs{\lambda_n}\abs{x\dual_\mathbb{R}(x_n e_n)}
\leqslant
\sum_{n\in F}\abs{x\dual_\mathbb{R}(x_n e_n)}
\]

En posant $\varepsilon_n=signe(x\dual_\mathbb{R}(x_n e_n))$, on a

\[
\sum_{n\in F}\abs{x\dual_\mathbb{R}(x_n e_n)}
=
\sum_{n\in F}x\dual_\mathbb{R}(\varepsilon_n x_n e_n)
=
\abs{x\dual_\mathbb{R}\left(\sum_{n\in F}\varepsilon_n x_n e_n\right)}
\leqslant
\norme{\sum_{n\in F}\varepsilon_n x_n e_n}
\]

Maintenant que l'on a fait dispara�tre $x\dual_\mathbb{R}$, on peut � nouveau consid�rer l'espace initial plut�t que son �quivalent r�el. Avec $F^{+}=\{n\in F|\varepsilon_n=1\}$ et $F^{-}=\{n\in F|\varepsilon_n=-1\}$, on a

\[
\norme{\sum_{n\in F}\varepsilon_n x_n e_n}
\leqslant
\norme{\sum_{n\in F^{+}} x_n e_n}+\norme{\sum_{n\in F^{-}} x_n e_n}
\leqslant
2K\norme{x}
\]

La derni�re in�galit� �tant donn�e par le fait que $(e_n)_{n\in\mathbb{N}}$ admet $K$ pour constante de base inconditionnelle. En rassemblant partie r�elle et partie imaginaire, on obtient l'in�galit� annonc�e.

(2)\\
Soient $A\subseteq B$ deux ensembles finis non vides de naturels, et une suite de scalaires $(x_n)_{n\in\mathbb{N}}$. On a

\[
\begin{array}{lllll}
\displaystyle
\triplenorme{\sum_{n\in A} x_n e_n}
&
\displaystyle
=
\sup\left\{
\norme{\sum_{n\in A\cap F} \lambda_n x_n e_n} \text{tel que } \emptyset\neq F\subseteq\mathbb{N}\text{ fini}, \abs{\lambda_n}\leqslant 1,\forall n\in F
\right\}
\\
&
\displaystyle
\leqslant
\sup\left\{
\norme{\sum_{n\in B\cap F} \lambda_n x_n e_n} \text{tel que } \emptyset\neq F\subseteq\mathbb{N}\text{ fini}, \abs{\lambda_n}\leqslant 1,\forall n\in F
\right\}
\\
&
\displaystyle
=
\triplenorme{\sum_{n\in B} x_n e_n}
\end{array}
\]

(3)\\
Soient $\lambda, \mu\in\mathbb{K}$ et $t\in[0,1]$.

On montre que la fonction est convexe. On a

\[
\triplenorme{x + (t\lambda + (1-t)\mu)y}
=
\triplenorme{tx + (1-t)x + (t\lambda + (1-t)\mu)y}
\leqslant
t\triplenorme{x+\lambda y} + (1-t)\triplenorme{x+\mu y}
\]

On v�rifie qu'elle ne d�pend pas de $\Arg(\lambda)$. En effet,
\begin{align*}
&
\triplenorme{x+\lambda y}
\displaystyle
=
\sup\left\{
\norme{\sum_{n\in A\cap F}\!\!\lambda_n z_n e_n + \lambda\!\!\sum_{k\in B\cap F}\!\!\lambda_n z_n e_n} \text{tel que }\emptyset\neq F\subseteq\mathbb{N}\text{ fini}, \forall n\in F, \abs{\lambda_n}\leqslant 1
\right\}
\\
&
\displaystyle
=
\sup\left\{
\norme{\sum_{n\in A\cap F}\!\lambda_n z_n e_n + \abs{\lambda} \sum_{n\in B\cap F}\!\e^{i\arg(\lambda)}\lambda_n z_n e_n} \text{tel que } \emptyset\neq F\subseteq\mathbb{N}\text{ fini}, \forall n\in F, \abs{\lambda_n}\leqslant 1
\right\}
\\
&
\displaystyle
=
\triplenorme{x+\abs{\lambda}y}
\end{align*}

Enfin, elle est croissante en $\abs{\lambda}$. Supposons par l'absurde que ce ne soit pas le cas, prenons $0\leqslant r<R$, tels que $\triplenorme{x+r\lambda y}>\triplenorme{x+R\lambda y}$. Alors, comme cette fonction est ind�pendante de $\Arg(\lambda)$, on a �galement que $\triplenorme{x-r\lambda y}>\triplenorme{x-R\lambda y}$. Ceci contredit le fait qu'elle est convexe.
\end{proof}

