\begin{Lemme}\label{constantebaseincondnorme}
Soit une suite $(e_n)_{n\in\mathbb{N}}$ de $E$, supposons qu'elle admet une constante de base inconditionnelle $K>0$. On d�finit sur $\SpanAdh((e_n)_{n\in\mathbb{N}})$ la norme

\[
\triplenorme{x}
=
\sup\left\{
\norme{\sum_{n\in A} \lambda_n x_n e_n} \text{tel que } \emptyset\neq A\subseteq\mathbb{N}\text{ fini}, \abs{\lambda_n}\leqslant 1,\forall n\in A
\right\}
\]

Alors

\begin{enumerate}[(1)]
\item pour tout $x\in\SpanAdh((e_n)_{n\in\mathbb{N}})$, $\norme{x}\leqslant\triplenorme{x}\leqslant 4K\norme{x}$
\item $(e_n)_{n\in\mathbb{N}}$ a une constante de base inconditionnelle qui vaut 1 pour $(E,\triplenorme{.})$
\item pour toute paire d'ensembles finis disjoints non vides de naturels $A$ et $B$, pour toute suite de scalaires $(z_n)_{n\in\mathbb{N}}$, si on note

\[
x=\sum_{n\in A} z_n e_n ~~~~~~ y=\sum_{n\in B} z_n e_n
\]

alors quelque soit $\lambda\in\mathbb{K}$, $\triplenorme{x+\lambda y}=\triplenorme{x+\abs{\lambda}y}$
\end{enumerate}
\end{Lemme}

\begin{proof}
(1)\\
Soit $x\in\SpanAdh((f_n)_{n\in\mathbb{N}})$. En prenant $A=\{1,...,n\}$, pour tout $k\in A$, $\varepsilon_k=1$, et en faisant tendre $n$ vers l'infini, on obtient la premi�re in�galit�. R�ciproquement, soient $x\in E$, $A\subseteq\mathbb{N}$ un ensemble fini non vide et $\abs{\lambda_n}\leqslant 1$ pour tout $n\in A$. On a

\[
\norme{\sum_{n\in A} \lambda_n x_n e_n}
\leqslant
\norme{\sum_{n\in A} \Re(\lambda_n) x_n e_n}
+
\norme{\sum_{n\in A} i\Im(\lambda_n) x_n e_n}
\]

Les deux sommes seront tra�t�es identiquement. Alors, on suppose pour simplifier les notations que $\varepsilon_k\in[-1,1]$, quelque soit $k\in A$. On consid�re $E_\mathbb{R}$ l'�quivalent r�el de $E$. Par le th�or�me de Hahn-Banach (\ref{HBF}) il existe $x\dual_\mathbb{R}\in S(E_\mathbb{R}\dual)$ tel que

\[
\norme{\sum_{n\in A}\lambda_n x_n e_n}
=
x\dual_\mathbb{R}\left(\sum_{n\in A}\lambda_n x_n e_n\right)
\]

Alors

\[
\norme{\sum_{n\in A} \lambda_n x_n e_n}
\leqslant
\sum_{n\in A}\abs{\lambda_n}\abs{x\dual_\mathbb{R}(x_n e_n)}
\leqslant
\left(\max_{n\in A}\abs{\lambda_n}\right) \sum_{n\in A}\abs{x\dual_\mathbb{R}(x_n e_n)}
\]

En posant $\varepsilon_n=signe(x\dual_\mathbb{R}(x_n e_n))$, et puisque $\max\limits_{n\in A}\abs{\lambda_n}\leqslant 1$, on a

\[
\left(\max_{n\in A}\abs{\lambda_n}\right) \sum_{n\in A}\abs{x\dual_\mathbb{R}(x_n e_n)}
\leqslant
\sum_{n\in A}x\dual_\mathbb{R}(\varepsilon_n x_n e_n)
\leqslant
\norme{x\dual_\mathbb{R}\left(\sum_{n\in A}\varepsilon_n x_n e_n\right)}
\leqslant
\norme{\sum_{n\in A}\varepsilon_n x_n e_n}
\]

Maintenant que l'on a fait dispara�tre $x\dual_\mathbb{R}$, on peut � nouveau consid�rer l'espace initial plut�t que son �quivalent r�el. Avec $A^{+}=\{n\in A|\varepsilon_n=1\}$ et $A^{-}=\{n\in A|\varepsilon_n=-1\}$, on a

\[
\norme{\sum_{n\in A}\varepsilon_n x_n e_n}
\leqslant
\norme{\sum_{n\in A^{+}} x_n e_n}+\norme{\sum_{n\in A^{-}} x_n e_n}
\leqslant
2K\norme{x}
\]

La derni�re in�galit� �tant donn�e par le fait que $(e_n)_{n\in\mathbb{N}}$ admet $K$ pour constante de base inconditionnelle. En rassemblant partie r�elle et partie imaginaire, on obtient l'in�galit� annonc�e.

(2)\\
Soient $A\subseteq B$ deux ensembles finis non vides de naturels, et une suite de scalaires $(x_n)_{n\in\mathbb{N}}$. On a

\[
\begin{array}{lllll}
\displaystyle
\triplenorme{\sum_{n\in A} x_n e_n}
&
\displaystyle
=
\max\left\{
\norme{\sum_{n\in C} \lambda_n x_n e_n} \text{tel que } \emptyset\neq C\subseteq A\text{ et } \forall n\in C, \abs{\lambda_n}\leqslant 1
\right\}
\\
&
\displaystyle
\leqslant
\max\left\{
\norme{\sum_{n\in C} \lambda_n x_n e_n} \text{tel que } \emptyset\neq C\subseteq B\text{ et } \forall n\in C, \abs{\lambda_n}\leqslant 1
\right\}
\\
&
\displaystyle
=
\triplenorme{\sum_{n\in B} x_n e_n}
\end{array}
\]

(3)\\
Soit $\lambda\in\mathbb{K}$. On a
\begin{align*}
&
\triplenorme{x+\lambda y}
\displaystyle
=
\max\left\{
\norme{\sum_{n\in A\cap C}\!\!\lambda_n z_n e_n + \lambda\!\!\sum_{k\in B\cap C}\!\!\lambda_n z_n e_n} \text{tel que }\emptyset\neq C\subseteq\mathbb{N}\text{ fini}, \forall n\in C, \abs{\lambda_n}\leqslant 1
\right\}
\\
&
\displaystyle
=
\max\left\{
\norme{\sum_{n\in A\cap C}\!\lambda_n z_n e_n + \abs{\lambda} \sum_{n\in B\cap C}\!\e^{i\arg(\lambda)}\lambda_n z_n e_n} \text{tel que } \emptyset\neq C\subseteq\mathbb{N}\text{ fini}, \forall n\in C, \abs{\lambda_n}\leqslant 1
\right\}
\\
&
\displaystyle
=
\triplenorme{x+\abs{\lambda}y}
\end{align*}
\end{proof}

