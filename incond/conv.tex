\chapter{Bases de Schauder inconditionnelles}

Dans ce chapitre, nous allons �tudier un type tr�s particulier de base de Schauder, mais toutefois tr�s puissant. Avant ceci, on va rappeler plusieurs notions indispensables, ainsi que certaines de leurs caract�ristiques.

\section{Convergence absolue, inconditionnelle}

\begin{Def}\label{convabs}
Soient $E$ un espace vectoriel norm�, et $(x_n)_{n\in\mathbb{N}}$ une suite de $E$. On dit que la s�rie $\sum\limits_{n\in\mathbb{N}}x_n$ converge absolument si la suite $\left(\sum\limits_{k=0}^n \norme{x_k}\right)_{n\in\mathbb{N}}$ est convergente.
\end{Def}

\begin{Thm}\label{banachserie}
Un espace vectoriel norm� est un espace de Banach si et seulement si toutes ses s�ries absolument convergentes sont convergentes.
\end{Thm}

\begin{Def}\label{convincond}
Soit $(x_n)_{n\in\mathbb{N}}$ une suite de $E$. Alors, on dit que la s�rie $\sum\limits_{n\in\mathbb{N}}x_n$ converge inconditionnellement si quelque soit $\sigma$ une permutation de $\mathbb{N}$, la s�rie $\sum\limits_{n\in\mathbb{N}}x_{\sigma(n)}$ est convergente.
\end{Def}

\begin{Rem}
Bien que la d�finition n'exige pas que la limite soit ind�pendante de la permutation choisie, il s'av�re que c'est toujours le cas.
\end{Rem}

Telle que d�finie ci-dessus, la convergence inconditionnelle d'une suite n'est pas toujours �vidente � �tablir. Il existe n�anmoins un crit�re fort utile qui permet de l'obtenir.

\begin{Prop}\label{absimpliqueincond}
Toute s�rie absolument convergente est inconditionnellement convergente. La r�ciproque est vraie si l'espace est de dimension finie.
\end{Prop}

Maintenant, citons une propri�t� qui nous permettra d'utiliser la convergence inconditionnelle sous des formes tr�s diverses.

\begin{Prop}\label{convincondequiv}
Soit une suite $(x_n)_{n\in\mathbb{N}}$ de $E$. Les assertions suivantes sont �quivalentes:

\begin{enumerate}[(1)]
\item $\sum\limits_{n\in\mathbb{N}} x_n$ converge inconditionnellement
\item il existe $x\in E$ v�rifiant $\forall\varepsilon>0,\exists F_0\subset\mathbb{N}$ fini tel que $\forall F_0\subseteq F\subset\mathbb{N}$ fini, $\norme{x-\sum\limits_{n\in F}x_n}<\varepsilon$
\item $\forall\varepsilon>0,\exists n_0>0$ tel que $\forall F\subset\mathbb{N}$ fini, $\min(F)>n_0\Rightarrow \norme{\sum\limits_{n\in F}x_n}<\varepsilon$
\item $\sum\limits_{n\in\mathbb{N}}y_n$ converge pour toute sous-suite $(y_n)_{n\in\mathbb{N}}$ de $(x_n)_{n\in\mathbb{N}}$
\item $\sum\limits_{n\in\mathbb{N}}\varepsilon_n x_n$ converge pour toute suite $(\varepsilon_n)_{n\in\mathbb{N}}$ � valeur dans $\{-1,1\}$
\item $\sum\limits_{n\in\mathbb{N}}\lambda_n x_n$ converge pour toute suite born�e de scalaires $(\lambda_n)_{n\in\mathbb{N}}$
\item $\lim\limits_{n\rightarrow +\infty}\sup\limits_{x\dual\in B(E\dual)}\sum\limits_{k\geqslant n}\abs{x\dual(x_k)} = 0$
\end{enumerate}
\end{Prop}

