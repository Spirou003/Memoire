\section{Espace quotient}

\begin{Def}\label{espaceortho}
Soient $E$ un espace vectoriel norm� et $F$ un sous-espace ferm� de $E$. On dit que l'espace vectoriel
\[
F\ortho=\{x\dual\in E\dual | x\dual(F)=\{0\}\}
\]
est l'espace orthogonal de $F$.
\end{Def}

\begin{Prop}\label{quotientbanach}
Soit $E$ un espace de Banach et $F$ un sous-espace ferm� de $E$. Alors l'espace vectoriel $E/F$ muni de la norme
\[
\begin{array}{lllll}
\norme{.}&:&E/F&\maps&\mathbb{R}^+\\
&:&x+F&\mapsto&\inf\limits_{f\in F}\norme{x+f}
\end{array}
\]
est un espace de Banach.
\end{Prop}

\begin{Prop}\label{ForthoisoEsurF}
Soit $E$ un espace de Banach et $F$ un sous-espace ferm� de $E$. Alors l'application
\[
\begin{array}{lllll}
\varphi&:&F\ortho&\maps&(E/F)\dual\\
&:&x\dual&\mapsto&\left(
    \begin{array}{lll}
    E/F&\maps&\mathbb{K}\\
    x+F&\mapsto&x\dual(x)
    \end{array}
\right)
\end{array}
\]
est une isom�trie.
\end{Prop}

\begin{proof}
En premier lieu, il faut s'assurer que $\varphi$ est bien d�finie, c'est-�-dire que pour tout $x\dual\in F\ortho$, $\varphi(x\dual)\in (E/F)\dual$. Soient $x\in E$, $f\in F$ et $x\dual\in F\ortho$. On a que $x\dual(x+f)=x\dual(x)$ car $x\dual$ s'annule sur $F$, donc l'image de $x+F$ par $\varphi(x\dual)$ ne d�pend pas du repr�sentant de classe choisi. La lin�arit� et la continuit� de $\varphi(x\dual)$ sont �videntes.

La lin�arit� de $\varphi$ est elle aussi �vidente.

Soient $x\dual\in F\ortho$, $x\in E$ et $f\in F$. On a

\[
\abs{\varphi(x\dual)(x+F)}
=
\abs{x\dual(x)}
=
\abs{x\dual(x+f)}
\leqslant
\norme{x\dual}\norme{x+f}.
\]

Ce qui implique que $\abs{\varphi(x\dual)(x+F)}\leqslant \norme{x\dual}\norme{x+F}$. Par cons�quent, $\norme{\varphi(x\dual)}\leqslant\norme{x\dual}$ et $\varphi$ est continue de norme inf�rieure � 1.

On montre que $\varphi$ est surjective. Soit $\overline{x\dual}\in(E/F)\dual$. On prend

\[
\begin{array}{lllll}
x\dual&:&E&\maps&\mathbb{K}\\
&:&x&\mapsto&\overline{x\dual}(x+F)
\end{array}
\]

La v�rification du fait que $\varphi(x\dual)=\overline{x\dual}$ est �vidente, donc $\varphi$ est surjective pour peu que le candidat choisi soit une application lin�aire continue qui s'annule sur F. La lin�arit� de $x\dual$ est �vidente, et quelque soit $f\in F$ on a que $x\dual(f)=\overline{x\dual}(f+F)=\overline{x\dual}(0+F)=0$. Il ne reste qu'� d�montrer qu'elle est continue. On montrera qu'elle est continue et de norme inf�rieure � $\norme{\overline{x\dual}}$, ceci nous donnera le caract�re isom�trique de $\varphi$ et par cons�quent son injectivit�. Soit $x\in E$. On a

\[
\abs{x\dual(x)}
=
\abs{\overline{x\dual}(x+F)}
\leqslant
\norme{\overline{x\dual}}\norme{x+F}
\leqslant
\norme{\overline{x\dual}}\norme{x}
\]
\end{proof}


