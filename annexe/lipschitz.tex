\begin{Prop}\label{lipschitz}
Soient $E$ un espace vectoriel norm�, $c\in\mathbb{K}_0$ et $x\dual\in E\dual$ non nul. Alors, en posant $H=\inv{x\dual}(c)$, on a
\[
\sup_{x\in H}\frac{\abs{x\dual(x)}}{\norme{x}}
=
\sup_{\substack{x\in E\\x\neq 0}}\frac{\abs{x\dual(x)}}{\norme{x}}
=
\norme{x\dual}
\]
\end{Prop}

\begin{proof}
Soient $x\in H$ et $\alpha$ un scalaire non nul. On a
\[
\sup_{x\in H}\frac{\abs{x\dual(x)}}{\norme{x}}
=
\sup_{x\in H}\frac{\abs{x\dual(x)}}{\norme{x}}\frac{\alpha}{\alpha}
=
\sup_{x\in H}\frac{\abs{x\dual(\alpha x)}}{\norme{\alpha x}}
=
\sup_{x\in\alpha H}\frac{\abs{x\dual(x)}}{\norme{x}}
\]

Puisque $H$ est le translat� d'un sous-espace de $E$ de codimension 1, on a que $\mathbb{K}_0 H$ et $\ker(x\dual)$ partitionnent $E$. On en d�duit que
\[
\sup_{x\in H}\frac{\abs{x\dual(x)}}{\norme{x}}
=
\sup_{\substack{x\in E\\x\neq 0}}\frac{\abs{x\dual(x)}}{\norme{x}}
\]

De plus,
\[
\sup_{\substack{x\in E\\x\neq 0}}\frac{\abs{x\dual(x)}}{\norme{x}}
=
\sup_{\substack{x\in S(E)\\ \alpha>0}}\frac{\abs{x\dual(\alpha x)}}{\norme{\alpha x}}
=
\sup_{\substack{x\in S(E)\\ \alpha>0}}\frac{\alpha \abs{x\dual(x)}}{\alpha \norme{x}}
=
\sup_{x\in S(E)}\abs{x\dual(x)}
=
\norme{x\dual}
\]
\end{proof}