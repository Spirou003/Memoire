
\begin{Lemme}[Lemme alg�brique]\label{lemmealgebrique}
Soient $E$ un espace vectoriel norm�, $n\in\mathbb{N}_0$ et des fonctionnelles lin�airement ind�pendantes $x\dual_1,...,x\dual_n\in E\dual$. Alors pour tout $x\dual\in E\dual$ les assertions suivantes sont �quivalentes:
\begin{enumerate}[(1)]
\item $x\dual\in\Span(x\dual_1,...,x\dual_n)$
\item $\bigcap\limits_{k=1}^n \ker(x\dual_k)\subseteq\ker(x\dual)$
\end{enumerate}
\end{Lemme}
\begin{proof}
~

(1) $\Rightarrow$ (2)\\
%
Puisque $x\dual\in\Span(x\dual_1,...,x\dual_n)$, il existe $a_1,...,a_n\in\mathbb{K}$ tels que $x\dual=\sum\limits_{k=1}^n a_k x\dual_k$. Soit $x\in\bigcap\limits_{k=1}^n \ker(x\dual_k)$. Alors
\[
x\dual(x)
=
\sum\limits_{k=1}^n a_k x\dual_k(x)
=
0
\]

(2) $\Rightarrow$ (1)\\
%
Soit l'application lin�aire
\[
\begin{array}{lllll}
\varphi&:&E&\maps&\mathbb{K}^{n+1}\\
&:&x&\mapsto&(x\dual(x),x\dual_1(x),...,x\dual_n(x))
\end{array}
\]
Son image est un sous-espace vectoriel ferm� de $\mathbb{K}^{n+1}$. Alors par le th�or�me de Hahn-Banach (\ref{HBC}), puisque $(1, 0, ..., 0)\notin \Image(\varphi)$, on peut s�parer $\{(1, 0, ..., 0)\}$ et $\Image(\varphi)$, autrement dit il existe $A\in (\mathbb{K}^{n+1})\dual$ et $\alpha>0$ tels que pour tout $x\in E$, on a
\[
\abs{A(\varphi(x))}<\alpha<\abs{A((1,0,...,0))}
\]
Cela signifie que $\Image(A\circ\varphi)$ est born�. Or c'est un espace vectoriel, donc il est r�duit � l'�l�ment nul. De plus, $A$ s'identifie � $(a,a_1,...,a_n)\in\mathbb{K}^{n+1}$, alors d'une part $A((1,0,...,0))=a\neq 0$ car $\alpha<\abs{A((1,0,...,0))}$, d'autre part pour tout $x\in E$ on a que
\[
A(\varphi(x))
=
a x\dual(x) + \sum_{k=1}^n a_k x\dual_k(x)
=
0
\]
Ce qui signifie que $x\dual$ est une combinaison lin�aire de $x\dual_1,...,x\dual_n$.
\end{proof}
