\section{Th�or�mes de Hahn-Banach}

\begin{Thm}\label{HBF}
Toutes les applications du type $ev_x$ atteignent leur norme sur la boule unit�. Plus formellement,

\[
\forall x\in E, \exists \: x\dual\in E\dual \text{ tel que }\norme{x\dual}=1\text{ et }x\dual(x)=\norme{x}
\]
\end{Thm}

%\begin{Thm}\label{HBP}\end{Thm}
%Soient $E$ un espace vectoriel, $G$ un sous-espace vectoriel de $E$, $f$ une forme lin�aire de $E$ et $N$ une semi-norme de $E$. Si de plus pour tout $x\in G$ on a la majoration $\abs{f(x)}\leqslant N(x)$, alors la forme lin�aire $f$ s'�tend en une forme lin�aire d�finie sur $E$ qui v�rifie �galement cette majoration, sur $E$ tout entier.

\begin{Thm}\label{HBA}
Dans un espace vectoriel norm�, toute fonctionnelle continue d'un sous-espace vectoriel s'�tend en une fonctionnelle continue de m�me norme sur l'espace tout entier.
\end{Thm}

Pour le th�or�me suivant, voir \cite[p. 29-31]{BEAUZAMY}.

\begin{Thm}
Soit $E$ un $\mathbb{R}$-espace vectoriel norm�. Soient $A\subseteq E$ convexe ferm� non vide, $B\subseteq E$ convexe compact non vide, tels que $A\cap B=\emptyset$. Alors il existe $x\dual\in E\dual$ et $\alpha\in\mathbb{R}$ tels que pour tout $a\in A, b\in B$,
\[x\dual(a)<\alpha<x\dual(b)\]
\end{Thm}

\begin{Corol}\label{HBC}
Soit $E$ un $\mathbb{K}$-espace vectoriel norm�. Soient $A$ et $B$ deux convexes ferm�s non vides disjoints de $E$ tels que l'un des deux est compact et $A$ est $\mathbb{K}$-sym�trique, c'est-�-dire que $\lambda A\subseteq A$, quelque soit $\lambda\in S(\mathbb{K})$. Alors il existe $x\dual\in E\dual$ et $\alpha>0$ tels que pour tout $a\in A, b\in B$,
\[\abs{x\dual(a)}<\alpha<\abs{x\dual(b)}\]
\end{Corol}
\begin{proof}
On consid�re $E$ un espace vectoriel r�el. Le th�or�me nous donne $x\dual\in E\dual$ et $\alpha\in\mathbb{R}$ tels que pour tout $a\in A, b\in B$,
\[\begin{array}{rrrrr}
x\dual(a)&<&\alpha &<&x\dual(b)\\
-x\dual(a)&>&-\alpha &>&-x\dual(b)\\
-x\dual(b)&<&-\alpha &<&-x\dual(a)
\end{array}\]
Ce qui indique que l'ordre des in�galit�s est ind�pendant du fait que ce soit $A$ ou $B$ le compact.

Supposons que $A$ est $\mathbb{R}$-sym�trique. Soit $a\in A$, on a que $-a\in A$, et donc $0\leqslant\abs{x\dual(a)}<\alpha$. Par cons�quent, $\alpha>0$. Soit $b\in B$, on a que $0<\alpha<x\dual(b)\leqslant\abs{x\dual(b)}$. On en d�duit alors le corollaire pour un $\mathbb{R}$-espace vectoriel norm�.

Consid�rons maintenant $E$ un $\mathbb{C}$-espace vectoriel norm�, et notons $E_\mathbb{R}$ son �quivalent r�el. Appliquons le corollaire sur cet espace, on obtient alors $x_\mathbb{R}\dual\in E_\mathbb{R}\dual$ et $\alpha>0$ tels que pour tout $a\in A, b\in B, \abs{x_\mathbb{R}\dual(a)}<\alpha<\abs{x_\mathbb{R}\dual(b)}$. Le corollaire n'est pas encore d�montr�, puisqu'on l'a pour $x_\mathbb{R}\dual$ qui est $\mathbb{R}$-lin�aire, et donc pas n�cessairement $\mathbb{C}$-lin�aire. Par contre, si on d�finit $x\dual$ tel que pour tout $x\in E$,
\[
x\dual(x)=x_\mathbb{R}\dual(x)-ix_\mathbb{R}\dual(ix)
\]
on obtient bien une application de $E\dual$. Montrons que les in�galit�s souhait�es sont v�rifi�es aussi pour $x\dual$.

Soient alors $a\in A$ et $b\in B$. On a que
\[\abs{x\dual(b)}\geqslant\abs{\Re(x\dual(b))}=\abs{x_\mathbb{R}\dual(b)}>\alpha\]
De plus, $A$ est $\mathbb{C}$-sym�trique, donc il existe $\theta\in\mathbb{R}$ tel que $x\dual(a\e^{i\theta})>0$ et $a\e^{i\theta}\in A$. Par cons�quent,
\[
\abs{x\dual(a)}=\abs{x\dual(a\e^{i\theta})}=\abs{x_\mathbb{R}\dual(a\e^{i\theta})}<\alpha
\]
\end{proof}

\begin{Thm}\label{HBH}
Soient $E$ un espace vectoriel norm�, $L\neq\emptyset$ un sous-espace affin de $E$ et $\Omega\neq\emptyset$ un ouvert convexe de $E$. Si $L$ et $\Omega$ sont disjoints, alors il existe un hyperplan affin qui contient $L$ et qui ne rencontre pas $\Omega$.
\end{Thm}
