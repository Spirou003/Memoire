\chapter{Bases de Schauder inconditionnelles}\label{chapter_schauder_incond}

Dans ce chapitre, nous allons �tudier un type tr�s particulier de base de Schauder, mais toutefois tr�s puissant. Avant ceci, on va rappeler plusieurs notions indispensables, ainsi que certaines de leurs caract�ristiques.

\chapter{Bases de Schauder inconditionnelles}

Dans ce chapitre, nous allons �tudier un type tr�s particulier de base de Schauder, mais toutefois tr�s puissant. Avant ceci, on va rappeler plusieurs notions indispensables, ainsi que certaines de leurs caract�ristiques.

\section{Convergence absolue, inconditionnelle}

\begin{Def}\label{convabs}
Soient $E$ un espace vectoriel norm�, et $(x_n)_{n\in\mathbb{N}}$ une suite de $E$. On dit que la s�rie $\sum\limits_{n\in\mathbb{N}}x_n$ converge absolument si la suite $\left(\sum\limits_{k=0}^n \norme{x_k}\right)_{n\in\mathbb{N}}$ est convergente.
\end{Def}

\begin{Thm}\label{banachserie}
Un espace vectoriel norm� est un espace de Banach si et seulement si toutes ses s�ries absolument convergentes sont convergentes.
\end{Thm}

\begin{Def}\label{convincond}
Soit $(x_n)_{n\in\mathbb{N}}$ une suite de $E$. Alors, on dit que la s�rie $\sum\limits_{n\in\mathbb{N}}x_n$ converge inconditionnellement si quelque soit $\sigma$ une permutation de $\mathbb{N}$, la s�rie $\sum\limits_{n\in\mathbb{N}}x_{\sigma(n)}$ est convergente.
\end{Def}

\begin{Rem}
Bien que la d�finition n'exige pas que la limite soit ind�pendante de la permutation choisie, il s'av�re que c'est toujours le cas.
\end{Rem}

Telle que d�finie ci-dessus, la convergence inconditionnelle d'une suite n'est pas toujours �vidente � �tablir. Il existe n�anmoins un crit�re fort utile qui permet de l'obtenir.

\begin{Prop}\label{absimpliqueincond}
Toute s�rie absolument convergente est inconditionnellement convergente. La r�ciproque est vraie si l'espace est de dimension finie.
\end{Prop}

Maintenant, citons une propri�t� qui nous permettra d'utiliser la convergence inconditionnelle sous des formes tr�s diverses.

\begin{Prop}\label{convincondequiv}
Soit une suite $(x_n)_{n\in\mathbb{N}}$ de $E$. Les assertions suivantes sont �quivalentes:

\begin{enumerate}[(1)]
\item $\sum\limits_{n\in\mathbb{N}} x_n$ converge inconditionnellement
\item Il existe $x\in E$ v�rifiant $\forall\varepsilon>0,\exists F_0\subset\mathbb{N}$ fini tel que $\forall F_0\subseteq F\subset\mathbb{N}$ fini, $\norme{x-\sum\limits_{n\in F}x_n}<\varepsilon$
\item $\forall\varepsilon>0,\exists n_0>0$ tel que $\forall F\subset\mathbb{N}$ fini, $\min(F)>n_0\Rightarrow \norme{\sum\limits_{n\in F}x_n}<\varepsilon$
\item $\sum\limits_{n\in\mathbb{N}}y_n$ converge pour toute sous-suite $(y_n)_{n\in\mathbb{N}}$ de $(x_n)_{n\in\mathbb{N}}$
\item $\sum\limits_{n\in\mathbb{N}}\varepsilon_n x_n$ converge pour toute suite $(\varepsilon_n)_{n\in\mathbb{N}}$ � valeur dans $\{-1,1\}$
\item $\sum\limits_{n\in\mathbb{N}}\lambda_n x_n$ converge pour toute suite born�e de scalaires $(\lambda_n)_{n\in\mathbb{N}}$
\item $\lim\limits_{n\rightarrow +\infty}\sup\limits_{x\dual\in B(E\dual)}\sum\limits_{k\geqslant n}\abs{x\dual(x_k)} = 0$
\end{enumerate}
\end{Prop}



\section{Base de Schauder inconditionnelle}

\begin{Def}
Soit $(e_n)_{n\in\mathbb{N}}$ une suite de $E$. On dit que $(e_n)_{n\in\mathbb{N}}$ est une suite basique inconditionnelle si c'est une suite basique pour laquelle chaque s�rie convergente $\sum\limits_{n\in\mathbb{N}}a_n e_n$ est inconditionnellement convergente. Si de plus, $\SpanAdh((e_n)_{n\in\mathbb{N}})=E$, alors on dit que c'est une base de Schauder inconditionnelle.
\end{Def}

Comme cons�quence imm�diate de la proposition (\ref{convincondequiv}), on a le r�sultat suivant:

\begin{Prop}\label{schauderincondequiv}
Soit une suite basique $(e_n)_{n\in\mathbb{N}}$ de $E$. Alors, les assertions suivantes sont �quivalentes:

\begin{enumerate}[(1)]
\item $(e_n)_{n\in\mathbb{N}}$ est une suite basique inconditionnelle
%
\item Quelque soit une s�rie convergente $\sum\limits_{n\in\mathbb{N}}a_n e_n=a$, et quelque soit $\varepsilon>0$, il existe $F_0\subset\mathbb{N}$ fini tel que pour tout ensemble fini $F_0\subseteq F\subset\mathbb{N}$, on a $\norme{a-\sum\limits_{n\in F}a_n e_n}<\varepsilon$
%
\item Quelque soit une s�rie convergente $\sum\limits_{n\in\mathbb{N}}a_n e_n$, on a $\forall\varepsilon>0,\exists n_0>0$ tel que $\forall F\subset\mathbb{N}$ fini, $\min(F)>n_0\Rightarrow\norme{\sum\limits_{n\in F}a_n e_n}<\varepsilon$
%
\item Quelque soit une s�rie convergente $\sum\limits_{n\in\mathbb{N}}a_n e_n$, la s�rie $\sum\limits_{n\in\mathbb{N}}y_n$ converge pour toute sous-suite $(y_n)_{n\in\mathbb{N}}$ de $(a_n e_n)_{n\in\mathbb{N}}$
%
\item Quelque soit une s�rie convergente $\sum\limits_{n\in\mathbb{N}}a_n e_n$, la s�rie $\sum\limits_{n\in\mathbb{N}}\varepsilon_n a_n e_n$ converge pour toute suite $(\varepsilon_n)_{n\in\mathbb{N}}$ � valeurs dans $\{-1,1\}$
%
\item Quelque soit une s�rie convergente $\sum\limits_{n\in\mathbb{N}}a_n e_n$, la s�rie $\sum\limits_{n\in\mathbb{N}}\lambda_n a_n e_n$ converge pour toute suite born�e de scalaires $(\lambda_n)_{n\in\mathbb{N}}$
%
\item Quelque soit une s�rie convergente $\sum\limits_{n\in\mathbb{N}}a_n e_n$, la s�rie $\sum\limits_{n\in\mathbb{N}}b_n e_n$ converge pour toute suite de scalaires $(b_n)_{n\in\mathbb{N}}$ major�e par $(a_n)_{n\in\mathbb{N}}$
%
\item Quelque soit une s�rie convergente $\sum\limits_{n\in\mathbb{N}}a_n e_n$, on a $\lim\limits_{n\rightarrow+\infty}\sup\limits_{x\dual\in B(E\dual)}\sum\limits_{k\geqslant n}\abs{x\dual(a_k e_k)}=0$
\end{enumerate}
\end{Prop}

\begin{Rem}
La propri�t� (7) n'est qu'une r��criture de la propri�t� (6), dans le cas particulier d'une suite de scalaires born�e par 1.
\end{Rem}

\begin{Corol}
La base canonique des espaces $c_0$ et $l^p$, pour $p\in[1,+\infty[$ est une base de Schauder inconditionnelle.
\end{Corol}
\begin{proof}En utilisant la propri�t� (5), c'est imm�diat.\end{proof}

\begin{Def}\label{constanteincond}
Soit une suite $(e_n)_{n\in\mathbb{N}}$ de vecteurs de $E$. On dit que cette suite admet une constante de base inconditionnelle s'il existe $K>0$ tel que pour toute paire d'ensembles finis non vides $A\subseteq B\subseteq\mathbb{N}$ et pour toute suite de scalaires $(a_n)_{n\in\mathbb{N}}$, on a
\[
\norme{\sum_{n\in A}a_n e_n}\leqslant K\norme{\sum_{n\in B}a_n e_n}
\]
Si $K>0$ est la plus petite valeur v�rifiant cette propri�t�, elle est appel�e la constante de base inconditionnelle de $(e_n)_{n\in\mathbb{N}}$.
\end{Def}

Comme pour les bases de Schauder, on se convainc ais�ment qu'une suite a toujours une constante de base inconditionnelle plus grande que 1, en prenant $A=B$.

\begin{Rem}
La d�finition d'une constante de base inconditionnelle implique d'une part que la suite est une suite basique (proposition (\ref{schauderbasique})), et d'autre part est �quivalente � dire que quelque soient deux ensembles $A\subseteq B\subseteq\mathbb{N}$ finis non vides, les projections
\[
P_{B,A}:\Span((e_n)_{n\in B})\maps\Span((e_n)_{n\in A})
\]
sont continues et de norme inf�rieure � $K$.

On en d�duit par ailleurs que quelque soient $A\subseteq B\subseteq\mathbb{N}$ non vides quelconques, les projections
\[
P_{B,A}:\SpanAdh((e_n)_{n\in B})\maps\SpanAdh((e_n)_{n\in A})
\]
sont �galement continues et de norme inf�rieure � $K$.
\end{Rem}

\begin{Prop}
Soit une suite $(e_n)_{n\in\mathbb{N}}$ de $E$. Les assertions suivantes sont �quivalentes:
\begin{enumerate}[(1)]
\item $(e_n)_{n\in\mathbb{N}}$ est une suite basique inconditionnelle
\item Pour toute permutation $\sigma$ de $\mathbb{N}$, $(e_{\sigma(n)})_{n\in\mathbb{N}}$ est une suite basique inconditionnelle
\item $(e_n)_{n\in\mathbb{N}}$ admet une constante de base inconditionnelle
\end{enumerate}
\end{Prop}

\begin{proof}
~

(1) $\Rightarrow$ (2)\\
%
Soient $\sigma$ une permutation de $\mathbb{N}$ et $x\in\SpanAdh((e_n)_{n\in\mathbb{N}})$. Par (1), il existe une suite de scalaires $(x_n)_{n\in\mathbb{N}}$ telle que la s�rie $\sum\limits_{n\in\mathbb{N}}x_n e_n$ converge inconditionnellement. Cela revient � dire que la s�rie $\sum\limits_{n\in\mathbb{N}}x_{\sigma(n)} e_{\sigma(n)}$ converge inconditionnellement. Donc $(e_{\sigma(n)})_{n\in\mathbb{N}}$ est une suite basique inconditionnelle.

(2) $\Rightarrow$ (1) est �vident.\newline

(1) $\Rightarrow$ (3)\\
Puisque $(e_n)_{n\in\mathbb{N}}$ est une suite basique, la projection $P_{B,A}$ est continue, quelque soient $A$ et $B$ deux ensembles finis non-vides tels que $A\subseteq B\subseteq\mathbb{N}$. On remarque que, si on trouve $K>0$ tel que pour toute suite de scalaires $(a_n)_{n\in\mathbb{N}}$, on ait

\begin{equation}
\norme{P_{\mathbb{N},A}}\leqslant K
\tag{*}
\end{equation}

alors on a �galement que 

\[
\norme{P_{B,A}}\leqslant K
\]

Par le th�or�me de Banach-Steinhaus, demander (*) pour tout ensemble fini $A\subseteq\mathbb{N}$ non vide est �quivalent � demander que pour tout $x\in\SpanAdh((e_n)_{n\in\mathbb{N}})$, 

\[
\sup\{\norme{P_{\mathbb{N},A}(x)}\text{ avec }\emptyset\neq A\subseteq\mathbb{N}\text{ fini}\}<+\infty
\]

Soit $x\in\SpanAdh((e_n)_{n\in\mathbb{N}})$. On utilise (3) dans la proposition (\ref{schauderincondequiv}). En y prenant $\varepsilon=1$, on obtient $n_0\in\mathbb{N}$ tel que pour tout ensemble fini $F\subseteq\mathbb{N}$ qui v�rifie $\min(F)>n_0$, on a

\[
\norme{\sum\limits_{n\in F}x_n e_n}<\varepsilon
\]

Soit $A\subseteq\mathbb{N}$ fini non vide. Si $\max(A)\leqslant n_0$, alors

\[
\norme{P_{\mathbb{N},A}(x)}
=
\norme{\sum_{n\in A}x_n e_n}
\leqslant
\sum_{n=0}^{n_0}\norme{x_n e_n}
\]

Si $\max(A)>n_0$, on a

\[
\norme{P_{\mathbb{N},A}(x)}
=
\norme{\sum_{n\in A}x_n e_n}
\leqslant
\norme{\sum_{\substack{n=0\\n\in A}}^{n_0}x_n e_n} + \norme{\sum_{\substack{n>n_0\\n\in A}}x_n e_n}
\leqslant
\sum_{n=0}^{n_0}\norme{x_n e_n} + \varepsilon
\]

Dans tous les cas, le dernier membre ne d�pend pas de $A$, ce que l'on cherchait � d�montrer.

(3) $\Rightarrow$ (1)\\
On d�montre (4) dans la proposition (\ref{schauderincondequiv}). Soient $x\in\SpanAdh((e_n)_{n\in\mathbb{N}})$, une suite strictement croissante de naturels $(n_k)_{k\in\mathbb{N}}$ et $\varepsilon>0$. Puisque $x\in\SpanAdh((e_n)_{n\in\mathbb{N}})$, il existe une suite de scalaires $(x_n)_{n\in\mathbb{N}}$ telle que la s�rie $\sum\limits_{n\in\mathbb{N}}x_n e_n$ converge vers $x$, et en prenant $\frac{\varepsilon}{K}$ on obtient $m_0$ tel que pour tout $n\geqslant m\geqslant m_0$,

\[
\norme{\sum_{k=m}^n x_k e_k}
<
\frac{\varepsilon}{K}
\]

Soient $n\geqslant m$ deux naturels tels que $m_0\leqslant n_m$. Alors on a

\[
\norme{\sum_{k=m}^n x_{n_k}e_{n_k}}
\leqslant
K\norme{\sum_{k=n_m}^{n_n}x_k e_k}
\leqslant
\varepsilon
\]

Ce qui indique que la suite $\left(\sum\limits_{k=0}^n x_{n_k}e_{n_k}\right)_{n\in\mathbb{N}}$ est de Cauchy, et donc convergente.
\end{proof}



\begin{Lemme}\label{constantebaseincondnorme}
Soit une suite $(e_n)_{n\in\mathbb{N}}$ de $E$, supposons qu'elle admet une constante de base inconditionnelle $K>0$. On d�finit sur $\SpanAdh((e_n)_{n\in\mathbb{N}})$ la norme

\[
\triplenorme{x}
=
\sup\left\{
\norme{\sum_{n\in A} \lambda_n x_n e_n} \text{tel que } \emptyset\neq A\subseteq\mathbb{N}\text{ fini}, \abs{\lambda_n}\leqslant 1,\forall n\in A
\right\}
\]

Alors

\begin{enumerate}[(1)]
\item pour tout $x\in\SpanAdh((e_n)_{n\in\mathbb{N}})$, $\norme{x}\leqslant\triplenorme{x}\leqslant 4K\norme{x}$
\item $(e_n)_{n\in\mathbb{N}}$ a une constante de base inconditionnelle qui vaut 1 pour $(E,\triplenorme{.})$
\item pour toute paire d'ensembles finis disjoints non vides de naturels $A$ et $B$, pour toute suite de scalaires $(z_n)_{n\in\mathbb{N}}$, si on note

\[
x=\sum_{n\in A} z_n e_n ~~~~~~ y=\sum_{n\in B} z_n e_n
\]

alors quelque soit $\lambda\in\mathbb{K}$, $\triplenorme{x+\lambda y}=\triplenorme{x+\abs{\lambda}y}$
\end{enumerate}
\end{Lemme}

\begin{proof}
(1)\\
Soit $x\in\SpanAdh((f_n)_{n\in\mathbb{N}})$. En prenant $A=\{1,...,n\}$, pour tout $k\in A$, $\varepsilon_k=1$, et en faisant tendre $n$ vers l'infini, on obtient la premi�re in�galit�. R�ciproquement, soient $x\in E$, $A\subseteq\mathbb{N}$ un ensemble fini non vide et $\abs{\lambda_n}\leqslant 1$ pour tout $n\in A$. On a

\[
\norme{\sum_{n\in A} \lambda_n x_n e_n}
\leqslant
\norme{\sum_{n\in A} \Re(\lambda_n) x_n e_n}
+
\norme{\sum_{n\in A} i\Im(\lambda_n) x_n e_n}
\]

Les deux sommes seront tra�t�es identiquement. Alors, on suppose pour simplifier les notations que $\varepsilon_k\in[-1,1]$, quelque soit $k\in A$. On consid�re $E_\mathbb{R}$ l'�quivalent r�el de $E$. Par le th�or�me de Hahn-Banach (\ref{HBF}) il existe $x\dual_\mathbb{R}\in S(E_\mathbb{R}\dual)$ tel que

\[
\norme{\sum_{n\in A}\lambda_n x_n e_n}
=
x\dual_\mathbb{R}\left(\sum_{n\in A}\lambda_n x_n e_n\right)
\]

Alors

\[
\norme{\sum_{n\in A} \lambda_n x_n e_n}
\leqslant
\sum_{n\in A}\abs{\lambda_n}\abs{x\dual_\mathbb{R}(x_n e_n)}
\leqslant
\left(\max_{n\in A}\abs{\lambda_n}\right) \sum_{n\in A}\abs{x\dual_\mathbb{R}(x_n e_n)}
\]

En posant $\varepsilon_n=signe(x\dual_\mathbb{R}(x_n e_n))$, et puisque $\max\limits_{n\in A}\abs{\lambda_n}\leqslant 1$, on a

\[
\left(\max_{n\in A}\abs{\lambda_n}\right) \sum_{n\in A}\abs{x\dual_\mathbb{R}(x_n e_n)}
\leqslant
\sum_{n\in A}x\dual_\mathbb{R}(\varepsilon_n x_n e_n)
\leqslant
\norme{x\dual_\mathbb{R}\left(\sum_{n\in A}\varepsilon_n x_n e_n\right)}
\leqslant
\norme{\sum_{n\in A}\varepsilon_n x_n e_n}
\]

Maintenant que l'on a fait dispara�tre $x\dual_\mathbb{R}$, on peut � nouveau consid�rer l'espace initial plut�t que son �quivalent r�el. Avec $A^{+}=\{n\in A|\varepsilon_n=1\}$ et $A^{-}=\{n\in A|\varepsilon_n=-1\}$, on a

\[
\norme{\sum_{n\in A}\varepsilon_n x_n e_n}
\leqslant
\norme{\sum_{n\in A^{+}} x_n e_n}+\norme{\sum_{n\in A^{-}} x_n e_n}
\leqslant
2K\norme{x}
\]

La derni�re in�galit� �tant donn�e par le fait que $(e_n)_{n\in\mathbb{N}}$ admet $K$ pour constante de base inconditionnelle. En rassemblant partie r�elle et partie imaginaire, on obtient l'in�galit� annonc�e.

(2)\\
Soient $A\subseteq B$ deux ensembles finis non vides de naturels, et une suite de scalaires $(x_n)_{n\in\mathbb{N}}$. On a

\[
\begin{array}{lllll}
\displaystyle
\triplenorme{\sum_{n\in A} x_n e_n}
&
\displaystyle
=
\max\left\{
\norme{\sum_{n\in C} \lambda_n x_n e_n} \text{tel que } \emptyset\neq C\subseteq A\text{ et } \forall n\in C, \abs{\lambda_n}\leqslant 1
\right\}
\\
&
\displaystyle
\leqslant
\max\left\{
\norme{\sum_{n\in C} \lambda_n x_n e_n} \text{tel que } \emptyset\neq C\subseteq B\text{ et } \forall n\in C, \abs{\lambda_n}\leqslant 1
\right\}
\\
&
\displaystyle
=
\triplenorme{\sum_{n\in B} x_n e_n}
\end{array}
\]

(3)\\
Soit $\lambda\in\mathbb{K}$. On a
\begin{align*}
&
\triplenorme{x+\lambda y}
\displaystyle
=
\max\left\{
\norme{\sum_{n\in A\cap C}\!\!\lambda_n z_n e_n + \lambda\!\!\sum_{k\in B\cap C}\!\!\lambda_n z_n e_n} \text{tel que }\emptyset\neq C\subseteq\mathbb{N}\text{ fini}, \forall n\in C, \abs{\lambda_n}\leqslant 1
\right\}
\\
&
\displaystyle
=
\max\left\{
\norme{\sum_{n\in A\cap C}\!\lambda_n z_n e_n + \abs{\lambda} \sum_{n\in B\cap C}\!\e^{i\arg(\lambda)}\lambda_n z_n e_n} \text{tel que } \emptyset\neq C\subseteq\mathbb{N}\text{ fini}, \forall n\in C, \abs{\lambda_n}\leqslant 1
\right\}
\\
&
\displaystyle
=
\triplenorme{x+\abs{\lambda}y}
\end{align*}
\end{proof}



\section{Caract�risation de sous-espaces, via les bases inconditionnelles}

\begin{Thm}\label{incondcompleteco}
Soit un espace de Banach $E$ contenant une suite basique inconditionnelle. Les assertions suivantes sont �quivalentes:
\begin{enumerate}[(1)]
\item Il existe une suite basique inconditionnelle non compl�te dans $E$
\item $E$ contient un sous-espace isomorphe � $c_0$
\end{enumerate}
\end{Thm}

\begin{proof}
~

(1) $\Rightarrow$ (2)\\
%
Soit $(e_n)_{n\in\mathbb{N}}$ une suite basique inconditionnelle non compl�te de $E$. On pose $E_0=\SpanAdh((e_n)_{n\in\mathbb{N}})$. On �quipe $E_0$ de la norme �quivalente $\triplenorme{.}$ d�crite par le lemme (\ref{constantebaseincondnorme}). Puisque $(e_n)_{n\in\mathbb{N}}$ n'est pas compl�te, il existe une suite de scalaires $(a_n)_{n\in\mathbb{N}}$ pour laquelle

\[
\sup_{n\in\mathbb{N}}\triplenorme{\sum_{k=0}^n a_k e_k}\leqslant 1
~~~~ \text{et} ~~~~
\left(\sum_{k=0}^n a_k e_k\right)_{n\in\mathbb{N}} \text{ ne converge pas}
\]

Par le (2) du lemme (\ref{constantebaseincondnorme}), on a �galement que

\begin{equation}
\text{pour tout ensemble fini non vide }A\subseteq\mathbb{N}, \triplenorme{\sum_{n\in A} a_n e_n}\leqslant 1
\label{incondcompletecoapplylemme}\tag{*}
\end{equation}

Puisque la s�rie n'est pas convergente, il existe $\theta>0$ tel que pour tout $n\in\mathbb{N}$, on trouve $N\geqslant M\geqslant n$ v�rifiant

\[
\triplenorme{\sum_{k=M}^N a_k e_k}\geqslant\theta
\]

Prenons $n=0$, on obtient $n_0\geqslant m_0\geqslant 0$ tels que

\[
\triplenorme{\sum_{k=m_0}^{n_0} a_k e_k}\geqslant\theta
\]

On prend $n=n_0+1$, on obtient $n_1\geqslant m_1\geqslant n_0+1$ qui v�rifient la m�me propri�t�. En reproduisant ce proc�d�, on construit deux suites de naturels $(n_k)_{k\in\mathbb{N}}$ et $(m_k)_{k\in\mathbb{N}}$ telles que

\[
m_0\leqslant n_0<m_1\leqslant n_1<...~m_k\leqslant n_k<m_{k+1}~...
~~~~ \text{et} ~~~~
1\geqslant\triplenorme{\sum_{l=m_k}^{n_k}a_l e_l}\geqslant\theta ~~\text{pour tout }k\in\mathbb{N}
\]

Posons $f_k=\sum\limits_{l=m_k}^{n_k}a_l e_l$, pour $k\in\mathbb{N}$. Soient $A\subseteq\mathbb{N}$ fini non vide et $(b_n)_{n\in\mathbb{N}}$ une suite de scalaires. Alors d'une part par (\ref{incondcompletecoapplylemme}) et le (3) du lemme (\ref{constantebaseincondnorme}) on a que

\[
\triplenorme{\sum_{n\in A}b_n f_n}
=
\triplenorme{\sum_{n\in A}\abs{b_n}f_n}
\leqslant
\max_{n\in A}\abs{b_n}\triplenorme{\sum_{n\in A}f_n}
\leqslant
\max_{n\in A}\abs{b_n}
\]

D'autre part, par d�finition de $\triplenorme{.}$ et des $(f_n)_{n\in\mathbb{N}}$, on a �galement

\[
\triplenorme{\sum_{n\in A}b_n f_n}
\geqslant
\max_{n\in A}\triplenorme{b_n f_n}
\geqslant
\theta\max_{n\in A}\abs{b_n}
\]

On en d�duit alors que pour toute s�rie convergente $\sum\limits_{n\in\mathbb{N}}b_n f_n$, on a

\[
\theta\sup_{n\in\mathbb{N}}\abs{b_n}
\leqslant
\triplenorme{\sum_{n\in\mathbb{N}}b_n f_n}
\leqslant
\sup_{n\in\mathbb{N}}\abs{b_n}
\]

Posons $F=\SpanAdh((f_n)_{n\in\mathbb{N}})$. Soit $f\in F$, supposons que la s�rie $\sum\limits_{n\in\mathbb{N}}b_n f_n$ converge vers $f$. On va montrer que $b_n\rightarrow 0$. Soit $\varepsilon>0$. Puisque la s�rie converge, avec $\varepsilon\theta$ on obtient $n_0$ tel que pour tout $n\geqslant m\geqslant n_0$, on a

\[
\triplenorme{\sum_{k=m}^n b_k f_k}
<
\varepsilon\theta
\]

Alors, on a �galement pour tout naturel $l$ compris entre $m$ et $n$ que

\[
\abs{b_l}
\leqslant
\max_{k=m}^n\abs{b_k}
\leqslant
\frac{1}{\theta}\triplenorme{\sum_{k=m}^n b_k f_k}
<
\varepsilon
\]

Donc $(b_n)_{n\in\mathbb{N}}\in c_0$. Alors, la correspondance $\sum\limits_{n\in\mathbb{N}}b_n f_n\longleftrightarrow(b_n)_{n\in\mathbb{N}}$ d�crit un isomorphisme de $F$ vers $c_0$.

(2) $\Rightarrow$ (1)\\
Notons $F$ le sous-espace de $E$ qui est isomorphe � $c_0$, et notons $T:c_0\maps F$ l'isomorphisme. La base canonique de l'espace $c_0$ �tant une base de Schauder inconditionnelle non compl�te, la suite $(T(e_n))_{n\in\mathbb{N}}$ est une base de Schauder inconditionnelle non compl�te de $F$, et donc est une suite basique inconditionnelle non compl�te de $E$.
\end{proof}

\begin{Thm}\label{incondcontractantelun}
Soit un espace de Banach contenant une suite basique inconditionnelle. Les assertions suivantes sont �quivalentes:
\begin{enumerate}[(1)]
\item Il existe une suite basique inconditionnelle non contractante dans $E$
\item $E$ contient un sous-espace isomorphe � $l^1$
\end{enumerate}
\end{Thm}

\begin{proof}
~

(1) $\Rightarrow$ (2)\\
%
Soit $(e_n)_{n\in\mathbb{N}}$ une suite basique inconditionnelle non contractante de $E$. On d�signe $\SpanAdh((e_n)_{n\in\mathbb{N}})$ par $E_0$ et on l'�quipe de la norme �quivalente $\triplenorme{.}$ d�crite par le lemme (\ref{constantebaseincondnorme}). Puisque la suite $(e_n)_{n\in\mathbb{N}}$ n'est pas contractante, il existe $x\dual\in S(E_0\dual)$ et $\theta>0$ tels que pour tout $n\in\mathbb{N}$, on trouve $N\geqslant n$ v�rifiant

\[
\sup
\left\{
\abs{x\dual(x)}
\left|
\triplenorme{x}=1,x=\sum_{k\geqslant N}x_k e_k\text{ et }(x_k)_{k\in\mathbb{N}}\in\mathbb{K}^\mathbb{N}
\right.
\right\}
>
\theta
\]

En prenant $n=0$, on obtient $N\geqslant 0$ et on peut trouver $x\in S(E_0)$ tel que

\[
\abs{x\dual(x)}>\theta
~~~~\text{et}~~~~
x=\sum_{k\geqslant N}x_k e_k 
~~~ \text{ avec }
(x_k)_{k\in\mathbb{N}}\text{ une suite de scalaires}
\]

Puisque la s�rie est convergente, il existe $M\geqslant N$ tel que

\[
\frac{1}{2}\theta
\leqslant
\abs{x\dual\left(\sum_{k=N}^M x_k e_k\right)}
\leqslant
\frac{3}{2}\theta
~~~~\text{et}~~~~
\frac{1}{2}
\leqslant
\triplenorme{\sum_{k=N}^M x_k e_k}
\leqslant
\frac{3}{2}
\]

Il existe alors un scalaire par lequel on peut multiplier cette somme partielle afin d'obtenir des scalaires $y_N,...,y_M$ tels que

\[
\frac{1}{3}\theta
\leqslant
x\dual\left(\sum_{k=N}^M y_k e_k\right)
\leqslant
3\theta
~~~~\text{et}~~~~
\triplenorme{\sum_{k=N}^M y_k e_k} = 1
\]

On note $f_0$ cette nouvelle somme partielle. On reproduit le processus en prenant $n=M+1$, on obtient $f_1\in E_0$, $L>M+1$ et des scalaires $y_{M+1},...,y_L$ qui satisfont aux propri�t�s

\[
\frac{1}{3}\theta
\leqslant
x\dual\left(\sum_{k=M+1}^L y_k e_k\right)
\leqslant
3\theta
~~~~~~~~
\triplenorme{\sum_{k=M+1}^L y_k e_k} = 1
~~~~\text{et}~~~~
f_1=\sum_{k=M+1}^L y_k e_k
\]

En it�rant ce processus, on obtient une suite $(f_n)_{n\in\mathbb{N}}$ de vecteurs de norme 1 et d'image par $x\dual$ sup�rieure � $\frac{\theta}{3}$. Alors, quelque soient un ensemble fini non vide $A\subseteq\mathbb{N}$ et une suite de scalaires $(a_n)_{n\in\mathbb{N}}$, on a d'une part

\[
\triplenorme{\sum_{n\in A}a_n f_n}
\leqslant
\sum_{n\in A}\triplenorme{a_n f_n}
=
\sum_{n\in A}\abs{a_n}
\]

et d'autre part par le (3) du lemme (\ref{constantebaseincondnorme})

\[
\triplenorme{\sum_{n\in A}a_n f_n}
=
\triplenorme{\sum_{n\in A}\abs{a_n} f_n}
\geqslant
\abs{x\dual\left(\sum_{n\in A}\abs{a_n} f_n\right)}
=
\sum_{n\in A}\abs{a_n}x\dual(f_n)
\geqslant
\frac{\theta}{3}\sum_{n\in A}\abs{a_n}
\]

Par cons�quent, si la s�rie $\sum\limits_{n\in\mathbb{N}}a_n f_n$ converge, alors $(a_n)_{n\in\mathbb{N}}\in l^1$. Dans ce cas, la correspondance $\sum\limits_{n\in\mathbb{N}}a_n f_n\longleftrightarrow(a_n)_{n\in\mathbb{N}}\in l^1$ d�crit un isomorphisme de $\SpanAdh((f_n)_{n\in\mathbb{N}})$ vers $l^1$.

(2) $\Rightarrow$ (1)\\
Notons $F$ le sous-espace de $E$ qui est isomorphe � $l^1$, et notons $T:l^1\maps F$ l'isomorphisme. La base canonique de l'espace $l^1$ �tant une base de Schauder inconditionnelle non contractante, la suite $(T(e_n))_{n\in\mathbb{N}}$ est une base de Schauder inconditionnelle non contractante de $F$, et donc est une suite basique inconditionnelle non contractante de $E$.
\end{proof}

De ces deux th�or�mes ainsi que du th�or�me de James (\ref{JamesSchauder}), on en d�duit le th�or�me suivant:

\begin{Thm}[James]\label{incondjames}
Soit un espace de Banach contenant une base de Schauder inconditionnelle $(e_n)_{n\in\mathbb{N}}$. Les assertions suivantes sont �quivalentes:
\begin{enumerate}[(1)]
\item $E$ est r�flexif
\item $E$ ne contient aucun sous-espace isomorphe � $c_0$ ou � $l^1$
\end{enumerate}
\end{Thm}

\begin{proof}
~

(1) $\Rightarrow$ (2)\\
%
Puisque $E$ est r�flexif, par la proposition (\ref{equivreflexif}) tout sous-espace de $E$ l'est aussi. Puisque ni $c_0$ ni $l^1$ n'est r�flexif, aucun des deux ne peut �tre isomorphe � un sous-espace de $E$.

(2) $\Rightarrow$ (1)\\
%
Puisque $E$ ne contient aucun espace isomorphe � $c_0$, toutes ses bases de Schauder inconditionnelle sont compl�tes. De plus, $E$ ne contient aucun sous-espace isomorphe � $l^1$, ce qui implique que toutes ses bases de Schauder inconditionnelle soient contractantes. Par hypoth�se, $E$ contient une base de Schauder inconditionnelle, elle est alors compl�te et contractante. On conclut avec le th�or�me (\ref{JamesSchauder}).
\end{proof}



\chapterref 

La section concernant la convergence inconditionnelle est bas�e sur l'ouvrage de Christopher Heil (voir \cite{HEIL}). Les r�sultats cit�s ont tous �t� d�montr�s lors de ma derni�re pr�sentation au cours d'{\textit{Analyse math�matique 4}}. Le reste du chapitre est inspir� � la fois du m�me ouvrage et de celui de Bernard Beauzamy (voir \cite[p. 88-95]{BEAUZAMY}).
